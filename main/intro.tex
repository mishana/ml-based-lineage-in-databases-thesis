\chapter{Introduction}
\label{chap:intro}
% do we need to add TOC lines?

%\begin{figure}
%  \centering
%  \includegraphics[width=0.75\textwidth]{main/graphics/a_blowup.pdf}
%  \caption{This is a caption}
%\end{figure}

\subsection*{Data Lineage}
The focus of this work is providing explanations (or justifications) for the existence of tuples in a Database Management System (DBMS). These explanations are also known as \textit{data provenance} \cite{cheney2009provenance}.
Provenance in the literature \cite{Ives2008, Deutch2015} often refers to forms of ``justifying'' query results. That is, the provenance context is the state of the database (DB) just before the query execution. The specific type of provenance on which we focus is \textit{lineage} \cite{Cui:2000:TLV:357775.357777}, namely a collection of DB tuples whose existence led to the existence of tuple \textit{t} in a query result. 


\subsection*{Distant Lineage}
We take a more comprehensive look. We track lineage of tuples throughout their existence, while distinguishing between tuples that are inserted explicitly and independently of DB content (these are our ``building blocks'') and tuples that are inserted via a query (or, more generally, that depend on the contents of the DB).
In a real-life setting - tuples that are inserted via a query can be one of the following:
\begin{itemize}
    \item A hierarchy of materialized views - where each view can depend both on the DB and on previously defined views.
    \item Tuples that are inserted via a SQL \texttt{INSERT INTO SELECT} statement.
    \item Tuples that are inserted via a SQL \texttt{UPDATE} statement.
    \item A query result with explicitly added data fields, that is added back to another table in the DB. For example, get names of customers retrieved from an \texttt{orders} table, calculate some non-database-resident ``psychological profile" for each customer and insert both to a \texttt{customer\_profile} table for future recommendations.
\end{itemize}


\subsection*{Approximate Lineage}
As time goes on, provenance information for tuples that are inserted via a query may become complex (e.g., by tracking semiring formulas, as presented in \cite{green2007provenance}, or circuits as presented  in \cite{Deutch2014, Senellart2017}). Thus, the goal of this work is providing ``simple to work with'' and useful approximate lineage (using ML and NLP techniques), while requiring only a constant additional space per tuple. This approximate lineage is compared against state of the art ``exact provenance tracking system'' in terms of explainability and maintainability.\\


\subsection*{Main Contributions}
\par The main contributions of this work are as follows:
\begin{enumerate}
    \item The efficient, with constant additional space per tuple, usage of word vectors to encode \textit{lifelong} lineage.
    \item A family of algorithms and enhancements that render the approach practical for both direct (i.e., current DB state) and indirect (i.e., all history) lineage computations.
    \item Thorough experimentation which exhibits high usefulness potential.
\end{enumerate}


\subsection*{Organization}
The rest of this work is organized as follows.
Previous work on provenance is surveyed in depth in Chapter 2.
In Chapter 3 we discuss relational embeddings and the motivation to use them for lineage encoding.
In Chapter 4 we introduce basic algorithms for creating per-tuple lineage encoding vectors.
In Chapter 5 we adapt these basic algorithms to per-column lineage encoding vectors.
Chapter 6 discusses improvements to the previously presented algorithms.
In Chapter 7 we present thorough experimental results.
In Chapter 8 we provide further discussion on a few topics presented in this paper.
We conclude in Chapter 9, in which we also outline directions for future research.

% \subsection*{An unnumbered subsection}

% You may want to break up the intro into parts with titles. Subsectioning without numbering is an option you might want to consider.

% Some people include a specific section overviewing the results ("In Chapter so-and-so, we will see how etc.") which is also a way of describing the structure of the thesis. But this is not necessary.

% \subsection*{Thesis options and appearance}

% Please note that the \texttt{iitthesis} class has several options when you use it, such as:
% \begin{itemize}
% \item \texttt{fullpageDraft} to avoid the margins necessary for proper binding when you make the final print
% \item \texttt{beforeDefense} makes the personal acknowledgements invisible; use this to print the copies you submit initially to the grad school for sending to the opponent panel, i.e. thesis readers (who shouldn't see those parts). For the final submission, after having successfully defended --- drop this option. 
% \item \texttt{noabbrevs} no notation \& abbreviations list will be included in the thesis.
% \end{itemize}

% \subsection*{Hebrew font}

% The \texttt{iitthesis} document class uses the David CLM font family for Hebrew text. CLM is a shorthand for ``Culmus'' (\texthebrew{קולמוס}) --- the name of a freely-available Hebrew font package. It may be bundled with your LaTeX distribution, or otherwise, must be available as system fonts. If you're missing the Culmus fonts, try adding an appropriate package from your LaTeX distribution or system distribution; alternatively, you might want to visit the Culmus project page at \url{http://culmus.sourceforge.net/} and download and install the fonts manually.

% \subsection*{Setting thesis meta-data and publication information}

% The document class used to generate this document defines several commands you can use to set information  regarding your thesis, which is used in the title pages and elsewhere in the front matter.  Every (or almost every) command has an English and a Hebrew variant, with a \texttt{English} or \texttt{Hebrew} suffix to the command name. Examples:
% \begin{itemize}
% \item \verb|\titleHebrew|, \verb|\titleEnglish|
% \item \verb|\authorHebrew|, \verb|\authorEnglish|
% \item \verb|\JewishDateHebrew|, \verb|\JewishDateEnglish|
% \item \verb|\GregorianDateHebrew|, \verb|\GregorianDateEnglish|
% \item \verb|\publicationinfoHebrew|, \verb|\publicationinfoEnglish|
% \end{itemize}

% The file \texttt{misc/thesis-fields.tex} contains invocations of several such commands (some of them commented-out with \texttt{\%}), and some additional information about them.

