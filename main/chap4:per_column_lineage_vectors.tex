\chapter{Per-Column Lineage Vectors}
\label{chap:per_column_lineage_vectors}


% \section{Per-Column Lineage Vectors}\label{sec:column-vectors}
\section{Rationale} Database relations do not behave like natural language text, and a sequence of columns values in a tuple does not usually construct a natural language sentence. Thus, building lineage vectors at the tuple level and comparing tuples on this basis - is noisy and potentially lossy in terms of information. Instead, we devise a genetics-influenced approach. Suppose a tuple represents the genetic code of an entity (a living organism in the DB world). The columns of an entity are its basic properties, i.e., the genes that combine to construct its genetic code. In this setting, querying the direct lineage of a result tuple is analogous to finding its predecessors through DNA testing. Following the approach presented above, we design an alternative lineage tracking mechanism, that of keeping track of and querying lineage (via embeddings) at the column (gene) level. This way, we manage to better distinguish between the provenance features and the textual characteristics of a tuple.\\ 

\section{Implementation} 
\begin{enumerate}
    \item Each tuple has a set of (up to \textit{max\_vectors\_num}) \textit{column lineage vectors} per-column, instead of a single set of tuple vectors (per-tuple). Let us denote the set of columns (features, characteristics) for which a tuple $t$ has lineage embeddings as $t.lineage\_columns$.
    \item From now on, we formally denote a column name, using the \textit{full name} notation, as $T.\texttt{Attr}$, such that $T$ is a relation name and \texttt{Attr} is a column name. This is meant to achieve better distinction at the lineage vectors level between tuples that originate from different relations with the same column names. 
    \item Given a tuple $t$, we say that $t.lineage\_columns = $
        \begin{equation*}
            t.native\_columns \cup t.distant\_columns,
        \end{equation*}
    such that $t.native\_columns$ is the set of columns that construct $t$ and $t.distant\_columns$ is the set of columns (features, characteristics) that $t$ ``inherited" from its predecessors, but are not reflected as data columns in $t$.
    Note that the same column name may be in $t$ and inherited from its predecessors as well. This issue will be dealt with shortly.\\
    In the following examples we use \texttt{A}, \texttt{B}, \texttt{C}, \texttt{D} to represent the full name of a column for brevity. For example, suppose a tuple $t \in T_{AB}$ (a table with only two columns - \texttt{A} and \texttt{B}) has the column lineage vectors $CV_t = \{\texttt{A}\!: LV_A,\: \texttt{B}\!: LV_B,\: \texttt{C}\!: LV_C,\: \texttt{D}\!: LV_D\}$ (a map of: \textit{columns $\rightarrow$ sets of lineage vectors}). We say that $t.native\_columns$ = $\{\texttt{A}, \texttt{B}\}$, $t.distant\_columns$ = $\{\texttt{C}, \texttt{D}\}$ and $t.lineage\_columns$ = $\{\texttt{A}, \texttt{B}, \texttt{C}, \texttt{D}\}$.
    \item When combining lineage embeddings (see Algorithms \ref{algo:add_col_approx},\ref{algo:mul_col_approx}), all calculations are done at the column level.
    \item After constructing a new tuple $t \in T$ via a query $q$ and its per-column lineage vectors $CV_t$ (via a series of + and $\cdot$ operations, see Algorithms \ref{algo:add_col_approx} and \ref{algo:mul_col_approx}, respectively) - special care must be taken in dealing with $native\_columns$ of $t$.
    That is, for every column $\texttt{A} \in t.native\_columns$ (\texttt{A} represents the full name of a column for brevity):
    % and $\texttt{A} \notin CV_t.columns$ (after either Algorithm \ref{algo:add_col_approx} or \ref{algo:mul_col_approx} was applied at the last step of constructing $CV_t$)
    \begin{enumerate}
        \item If $\texttt{A} \notin CV_t.columns$ and $t.\texttt{A}$ is set to an existing value from some column in the DB, e.g., $q$ = \texttt{(INSERT INTO T SELECT A' FROM ...)} then we set $CV_t[\texttt{A}] = CV_t[\texttt{A'}]$.
        \item If $\texttt{A} \notin CV_t.columns$ and $t.\texttt{A}$ is set to some constant value, e.g., $q$ = \texttt{(INSERT INTO T SELECT \textit{const} FROM ...)} then we set $CV_t[\texttt{A}]$ = $\{initial\_vector($ $const)\}$, such that $initial\_vector(const)$ is calculated according to Algorithm \ref{algo:init_approx} with the ``textified" (see section \ref{sec:latent_wv_model}) form of \textit{const} as input data.
        \item If $\texttt{A} \in CV_t.columns$ and $t.\texttt{A}$ is set to some constant value, e.g., $q$ = \texttt{(INSERT INTO T SELECT \textit{const} FROM ...)} then we set $CV_t[\texttt{A}]$ = $CV_t[\texttt{A}]\; \cdot \; \{initial\_$ $vector(const)\}$. This way, we incorporate both the existing lineage data and the newly calculated one for the column \texttt{A}.
    \end{enumerate}
    % \footnotetext{In the following examples we use the notation \texttt{A}, \texttt{B}, \texttt{C}, \texttt{D} to represent the full name of a column for brevity.}
    % For example, if $t_1$ has lineage vectors for columns \texttt{A},\texttt{B} and $t_2$ has lineage vectors for the columns \texttt{B},\texttt{C} then $t_1 + t_2$ will have lineage vectors $t_1.\texttt{A}$, $(t_1.\texttt{B} + t_2.\texttt{B})$\footnotemark, $t_2.\texttt{C}$ for the columns \texttt{A},\texttt{B},\texttt{C}, respectively. 
    \item When comparing a tuple $t$ (and its lineage embeddings) to a set of other tuples $T'$:
    \begin{enumerate}
        \item If $t'.lineage\_columns \nsubseteq t.lineage\_columns$ ($t' \in T'$) then $t'$ is definitely not a part of the lineage of $t$. Otherwise, all of the ``genes" (columns) of $t'$ would be reflected in $t$.
        \item The similarity between $t$ and $t' \in T'$ is averaged over the pair-wise similarities of their respective mutual genes (lineage columns).
        For example, say a tuple $t$ that has lineage vectors for columns \texttt{A},\texttt{B},\texttt{C} is compared to another tuple $t'$ that has lineage vectors for columns \texttt{B},\texttt{C},\texttt{D}:\\
        \begin{equation*}
            \operatorname{sim}(t, t') = avg(\{\operatorname{sim}(t.\texttt{B}, t'.\texttt{B}), 
            \operatorname{sim}(t.\texttt{C}, t'.\texttt{C})\})\\
        \end{equation*}
        where $avg$ is a function that returns the average of a collection of numbers. $\operatorname{sim}(t.\texttt{B}, t'.\texttt{B})$ is the calculated similarity between the lineage vectors for column \texttt{B} of $t$ and $t'$ (similarly for column \texttt{C}). Observe that \texttt{A} and \texttt{D} are not mutual ``genes" of the tuples $t$ and $t'$, and, thus they do not hold lineage information that is valuable to the similarity calculation.
    \end{enumerate}

\end{enumerate}
% \footnotetext{Here, $(t_1.\texttt{B} + t_2.\texttt{B})$ means applying Algorithm \ref{algo:add_approx} on the respective column lineage vectors.}

\section{Notes} 
\begin{itemize}
    \item The rationale behind this method is partially inspired by the construction of ``tuple vectors" by means of averaging the ``sentence vectors" of the columns (as discussed in section \ref{sec:latent_wv_model}).
    % \item Given a tuple $t$, we say that $t.lineage\_columns = $
    %     \begin{equation*}
    %         t.native\_columns \cup t.distant\_columns,
    %     \end{equation*}
    % such that $t.native\_columns$ is the set of columns that construct $t$ and $t.distant\_columns$ is the set of columns (features, characteristics) that $t$ ``inherited" from its predecessors, but are not reflected as data columns in $t$.
    % Note that the same column name may be in $t$ and inherited from its predecessors as well. This issue will be dealt with shortly.\\
    % In the following examples we use \texttt{A}, \texttt{B}, \texttt{C}, \texttt{D} to represent the full name of a column for brevity. For example, suppose a tuple $t \in T_{AB}$ (a table with only two columns - \texttt{A} and \texttt{B}) has the column lineage vectors $CV_t = \{\texttt{A}\!: LV_A,\: \texttt{B}\!: LV_B,\: \texttt{C}\!: LV_C,\: \texttt{D}\!: LV_D\}$ (a map of: \textit{columns $\rightarrow$ sets of lineage vectors}). We say that $t.native\_columns$ = $\{\texttt{A}, \texttt{B}\}$, $t.distant\_columns$ = $\{\texttt{C}, \texttt{D}\}$ and $t.lineage\_columns$ = $\{\texttt{A}, \texttt{B}, \texttt{C}, \texttt{D}\}$.
    % \item After constructing a new tuple $t \in T$ via a query $q$ and its per-column lineage vectors $CV_t$ (via a series of + and $\cdot$ operations, see Algorithms \ref{algo:add_col_approx} and \ref{algo:mul_col_approx}, respectively) - special care must be taken in dealing with $native\_columns$ of $t$.
    % That is, for every column $\texttt{A} \in t.native\_columns$ (\texttt{A} represents the full name of a column for brevity):
    % % and $\texttt{A} \notin CV_t.columns$ (after either Algorithm \ref{algo:add_col_approx} or \ref{algo:mul_col_approx} was applied at the last step of constructing $CV_t$)
    % \begin{enumerate}
    %     \item If $\texttt{A} \notin CV_t.columns$ and $t.\texttt{A}$ is set to an existing value from some column in the DB, e.g., $q$ = \texttt{(INSERT INTO T SELECT A' FROM ...)} then we set $CV_t[\texttt{A}] = CV_t[\texttt{A'}]$.
    %     \item If $\texttt{A} \notin CV_t.columns$ and $t.\texttt{A}$ is set to some constant value, e.g., $q$ = \texttt{(INSERT INTO T SELECT \textit{const} FROM ...)} then we set $CV_t[\texttt{A}]$ = $\{initial\_vector($ $const)\}$, such that $initial\_vector(const)$ is calculated according to Algorithm \ref{algo:init_approx} with the ``textified" (see section \ref{sec:latent_wv_model}) form of \textit{const} as input data.
    %     \item If $\texttt{A} \in CV_t.columns$ and $t.\texttt{A}$ is set to some constant value, e.g., $q$ = \texttt{(INSERT INTO T SELECT \textit{const} FROM ...)} then we set $CV_t[\texttt{A}]$ = $CV_t[\texttt{A}]\; \cdot \; \{initial\_$ $vector(const)\}$. This way, we incorporate both the existing lineage data and the newly calculated one for the column \texttt{A}.
    % \end{enumerate}
    \item $t.lineage\_columns$ (for a tuple $t$) might get large and include (almost) all the columns in the DB. This can render this solution impractical, since large modern systems may operate with hundreds and even more columns across the DB. There are various practical techniques that aim to limit the number of lineage columns per tuple:
    \begin{enumerate}
        \item Set a bound (hyperparameter) on the number of $lineage\_columns$.
        \item Give native priority. That is, prefer keeping $native$\mbox{-} $\_columns$ and cutting-off $distant\_columns$, whose influence on \textit{this} tuple is more remote, when the bound is reached (see example \ref{example:generational_priority} below).
        \item Remove the relation name prefix from a column name, i.e., consider all $T.\texttt{Attr}$ as simply \texttt{Attr}. This might result in a loss of information, but hopefully not too harmful to the overall performance of the method.
    \end{enumerate}
    
\end{itemize}
\begin{figure}[h]
\begin{algorithm2e}[H]
\SetAlgoLined
% \SetKwInOut{Input}{input}\SetKwInOut{Output}{output}
\SetKwFunction{ClusterVectorsUsingKMeans}{\fncolor{ClusterVectorsUsingKMeans}}
\SetKwData{pv}{CV$_3$}
\SetKwInOut{HP}{Hyper-Parameters}

% \HP{\textit{max\_vectors\_num} - maximum number of vectors per-tuple}
\KwIn{two maps of: \textit{columns $\rightarrow$ sets of lineage vectors} $CV_1, CV_2$ that represent the lineage of two tuples $t_1$ and $t_2$, respectively}
\KwResult{a new map of: \textit{columns $\rightarrow$ sets of lineage vectors} \pv, that represents $t_1 + t_2$}
\BlankLine
\BlankLine
 \pv = \{\} $\rightarrow$ \{\}\cmtcolor{\tcp*{an empty map of: \textit{columns $\rightarrow$ sets of lineage vectors}}}
 \ForEach{column $c \in (CV_1.columns \cup CV_2.columns)$}{
    \cmtcolor{\tcc{$CV_i.columns$ is the domain and $CV_i[c]$ is the corresponding \textit{set of lineage vectors} of tuple $t_i$ for the column $c$}}
    \uIf{$c \in (CV_1.columns \cap CV_2.columns)$}{
        $\pv[c]$ = $\paramcolor{CV_1[c]} \:\fncolor{\boldsymbol+}\: \paramcolor{CV_2[c]}$\cmtcolor{\tcp*{see Algorithm \ref{algo:add_approx}}}
        % \cmtcolor{\tcp*{an inline call to  $\boldsymbol+$ operation defined in Algorithm \ref{algo:add_approx}}}
    }
    \uElseIf{$c \in CV_1.columns$}{
        $\pv[c]$ = $CV_1[c]$\;
    }
    \Else{
        $\pv[c]$ = $CV_2[c]$\;
    }
 }
 \end{algorithm2e}
\caption{Column Lineage vectors: $\boldsymbol+$ \small{(addition) Algorithm}}\label{algo:add_col_approx}
\end{figure}
\begin{figure}[h]
\begin{algorithm2e}[H]
\SetAlgoLined
% \DontPrintSemicolon
% \SetKwInOut{Input}{input}\SetKwInOut{Output}{output}
\SetKwFunction{ClusterVectorsUsingKMeans}{\fncolor{ClusterVectorsUsingKMeans}}
\SetKwFunction{Avg}{\fncolor{Avg}}
\SetKwFunction{CartesianProduct}{\fncolor{CartesianProduct}}
\SetKwData{pv}{CV$_3$}
\SetKwInOut{HP}{Hyper-Parameters}

% \HP{\textit{max\_vectors\_num} - maximum number of vectors per-tuple}
\KwIn{two maps of: \textit{columns $\rightarrow$ sets of lineage vectors} $CV_1, CV_2$ that represent the lineage of two tuples $t_1$ and $t_2$, respectively}
\KwResult{a new map of: \textit{columns $\rightarrow$ sets of lineage vectors} \pv, that represents $t_1 \cdot t_2$}
\BlankLine
\BlankLine
 \pv = \{\} $\rightarrow$ \{\}\cmtcolor{\tcp*{an empty map of: \textit{columns $\rightarrow$ sets of lineage vectors}}}
 \ForEach{column $c \in (CV_1.columns \cup CV_2.columns)$}{
    \cmtcolor{\tcc{$CV_i.columns$ is the domain and $CV_i[c]$ is the corresponding \textit{set of lineage vectors} of tuple $t_i$ for the column $c \in CV_i.columns$}}
    \uIf{$c \in (CV_1.columns \cap CV_2.columns)$}{
        $\pv[c]$ = $\paramcolor{CV_1[c]} \:\fncolor{\boldsymbol\cdot}\: \paramcolor{CV_2[c]}$\cmtcolor{\tcp*{see Algorithm \ref{algo:mul_approx}}}
        % \cmtcolor{\tcp*{an inline call to $\boldsymbol\cdot$ operation defined in Algorithm \ref{algo:mul_approx}}}
    }
    \uElseIf{$c \in CV_1.columns$}{
        $\pv[c]$ = $CV_1[c]$\;
    }
    \Else{
        $\pv[c]$ = $CV_2[c]$\;
    }
 }
\end{algorithm2e}
\caption{Column Lineage vectors: $\boldsymbol\cdot$ \small{(multiplication) Algorithm}}\label{algo:mul_col_approx}
\end{figure}

\begin{runexample}
    % Show examples of addition and multiplication constructions
    Next, we show examples of the addition and multiplication column lineage vectors constructions (Algorithms 4 and 5, respectively).
    Suppose we have two tuples $t_1, t_2$ with respective maps of \textit{columns $\rightarrow$ sets of lineage vectors} $CV_1, CV_2$, such that:
    \begin{equation*}
        CV_1 = \{\texttt{A}\!: LV_A,\: \texttt{B}\!: LV_{B_1}\},\: CV_2 = \{\texttt{B}\!: LV_{B_2},\: \texttt{C}\!: LV_C\}
    \end{equation*}
    $\texttt{A}, \texttt{B}, \texttt{C}$ are full column names and $LV_A, LV_{B_1}, LV_{B_2}, LV_C$ are sets of lineage vectors.
\end{runexample}
% addition
\begin{example-withrun}\label{example:algo_add_col}
    Let us follow the construction of $CV_3$, which represents the column lineage embeddings of $t_1 + t_2$ using Algorithm \ref{algo:add_col_approx} (corresponds to alternative use of data, i.e., OR in the query):
    \begin{enumerate}
        \item $CV_3 = \{\} \rightarrow \{\}$
        \item $CV_1.columns \,\cup\, CV_2.columns = \{\texttt{A}, \texttt{B}\} \cup \{\texttt{B}, \texttt{C}\} = \{\texttt{A}, \texttt{B}, \texttt{C}\}$
        \item $\texttt{A} \in CV_1.columns \wedge \texttt{A} \notin CV_2.columns $\\ $\Rightarrow CV_3[\texttt{A}] = CV_1[\texttt{A}] = LV_A$
        \item $\texttt{B} \in (CV_1.columns \cap CV_2.columns) $\\ $\Rightarrow CV_3[\texttt{B}] = CV_1[\texttt{B}] + CV_2[\texttt{B}] = LV_{B_1} + LV_{B_2}$
        \item $\texttt{C} \notin CV_1.columns \wedge \texttt{C} \in CV_2.columns $\\ $\Rightarrow CV_3[\texttt{C}] = CV_2[\texttt{C}] = LV_C$
        \item $CV_3 = \{\texttt{A}\!: LV_A,\: \texttt{B}\!: LV_{B_1} + LV_{B_2},\: \texttt{C}\!: LV_C\}$
    \end{enumerate}
\end{example-withrun}
% multiplication
\begin{example-withrun}
    Let us follow the construction of $CV_3$, that represents the column lineage embeddings of $t_1 \cdot t_2$ using Algorithm \ref{algo:mul_col_approx} (corresponds to joint use of data, i.e., AND in the query):
    \begin{enumerate}
        \item $CV_3 = \{\} \rightarrow \{\}$
        \item $CV_1.columns \,\cup\, CV_2.columns = \{\texttt{A}, \texttt{B}\} \cup \{\texttt{B}, \texttt{C}\} = \{\texttt{A}, \texttt{B}, \texttt{C}\}$
        \item $\texttt{A} \in CV_1.columns \wedge \texttt{A} \notin CV_2.columns $\\ $\Rightarrow CV_3[\texttt{A}] = CV_1[\texttt{A}] = LV_A$
        \item $\texttt{B} \in (CV_1.columns \cap CV_2.columns) $\\ $\Rightarrow CV_3[\texttt{B}] = CV_1[\texttt{B}] \cdot CV_2[\texttt{B}] = LV_{B_1} \cdot LV_{B_2}$
        \item $\texttt{C} \notin CV_1.columns \wedge \texttt{C} \in CV_2.columns $\\ $\Rightarrow CV_3[\texttt{C}] = CV_2[\texttt{C}] = LV_C$
        \item $CV_3 = \{\texttt{A}\!: LV_A,\: \texttt{B}\!: LV_{B_1} \cdot LV_{B_2},\: \texttt{C}\!: LV_C\}$
    \end{enumerate}
\end{example-withrun}
% generational priority
\begin{example-withrun}\label{example:generational_priority}
    Let us follow the construction of $CV_3$, that represents the column lineage embeddings of $t_3 = t_1 + t_2$, such that the result is a new tuple $t_3 \in T_{AB}$ (a table with only two columns - \texttt{A} and \texttt{B}). In addition, we use a bound of $b = 2$ on the number of retained $lineage\_columns$ and implement native priority:
    \begin{enumerate}
        \item (see exapmle \ref{example:algo_add_col})\\ $CV_3 = \{\texttt{A}\!: LV_A,\: \texttt{B}\!: LV_{B_1} + LV_{B_2},\: \texttt{C}\!: LV_C\}$\\ $\Rightarrow t_3.lineage\_columns = \{\texttt{A}, \texttt{B}, \texttt{C}\}$
        \item $t_3 \in T_{AB}$ $\Rightarrow t_3.native\_columns = \{\texttt{A}, \texttt{B}\}$, \\$t_3.distant\_columns = \{\texttt{C}\}$
        \item $|t_3.lineage\_columns| = 3 > 2 = b$
        \item $CV_3 = \{\texttt{A}\!: LV_A,\: \texttt{B}\!: LV_{B_1} + LV_{B_2}$\}\\ (the \textit{distant\_column} \texttt{C} was eliminated)
    \end{enumerate}
\end{example-withrun}
