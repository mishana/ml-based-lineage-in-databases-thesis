\begin{table*}
\centering
\pgfplotstabletypeset[
    color cells={min=0.45,max=1.0}, format index, format table, format manufacturer, format method,
    col sep=comma,
    /pgfplots/colormap={orange}{
        rgb255(0cm)=(255,245,235);
        rgb255(1cm)=(253,141,60)},
    assign column name/.code=\pgfkeyssetvalue{/pgfplots/table/column
        name}{{\textbf{#1}}},
]{
    index,  Provenance Method,       manufacturer, {Lin. size},  {1.00$^{Top}$},            {0.75$^{Top}$},        {0.50$^{Top}$},        {0.25$^{Top}$}
    0,      Tuple Vectors,   general mills sales, 1966,           0.457782 ,        0.486102 ,     0.54878 ,       0.658537
    1,      Column Vectors (CV),   general mills sales, 1966,          0.806205 ,        0.911864 ,     0.961382 ,       1.0
    2,      CV + Bloom filters,   general mills sales, 1966,          1.0 ,        1.0 ,     1.0 ,       1.0
}
\textbf{\caption{\label{tab:3}
                A comparison of different approximate lineage computation methods on a single query-result tuple (\textit{general mills sales}). Lin. size is the size of the exact lineage for a result tuple (containing only tuples from the \texttt{serving\_size} table). $p^{Top}$ shows the precision of various methods for the top $p\cdot|Lineage|$ tuples in the approximate lineage. For example, 0.50$^{Top}$ is the precision for the top 983 ($=0.5\cdot1996$) tuples in the approximate lineage.
                }
        }
\end{table*}