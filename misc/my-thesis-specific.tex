% This file should contain your own definitions specific
% only to this thesis;

% What it contains below is used 
% simply for generating dummy text in the sample 
% content provided with the template (see mainchap1.tex);
% so you can safely delete this when creating your own
% thesis

\usepackage{centernot}
\usepackage{adjustbox}
\usepackage{float}
\usepackage{pgfplotstable}
\pgfplotstableset{
    /color cells/min/.initial=0,
    /color cells/max/.initial=1000,
    /color cells/textcolor/.initial=,
    %
    % Usage: 'color cells={min=<value which is mapped to lowest color>, 
    %   max = <value which is mapped to largest>}
    color cells/.code={%
        \pgfqkeys{/color cells}{#1}%
        \pgfkeysalso{%
            postproc cell content/.code={%
                %
                \begingroup
                %
                % acquire the value before any number printer changed
                % it:
                \pgfkeysgetvalue{/pgfplots/table/@preprocessed
                  cell content}\value
                \ifx\value\empty
                    \endgroup
                \else
                \pgfmathfloatparsenumber{\value}%
                \pgfmathfloattofixed{\pgfmathresult}%
                \let\value=\pgfmathresult
                %
                % map that value:
                \pgfplotscolormapaccess
                    [\pgfkeysvalueof{/color cells/min}:\pgfkeysvalueof{/color
                      cells/max}]
                    {\value}
                    {\pgfkeysvalueof{/pgfplots/colormap name}}%
                % now, \pgfmathresult contains {<R>,<G>,<B>}
                % 
                % acquire the value AFTER any preprocessor or
                % typesetter (like number printer) worked on it:
                \pgfkeysgetvalue{/pgfplots/table/@cell content}\typesetvalue
                \pgfkeysgetvalue{/color cells/textcolor}\textcolorvalue
                %
                % tex-expansion control
                % see http://tex.stackexchange.com/questions/12668/
                \toks0=\expandafter{\typesetvalue}%
                \xdef\temp{%
                    \noexpand\pgfkeysalso{%
                        @cell content={%
                            \noexpand\cellcolor[rgb]{\pgfmathresult}%
                            \noexpand\definecolor{mapped
                              color}{rgb}{\pgfmathresult}%
                            \ifx\textcolorvalue\empty
                            \else
                                \noexpand\color{\textcolorvalue}%
                            \fi
                            \the\toks0 %
                        }%
                    }%
                }%
                \endgroup
                \temp
                \fi
            }%
        }%
    },
    format index/.style={
        columns/index/.style={
            column name={},
            string type,
            postproc cell content/.code={},
            postproc cell  content/.append style={
            @cell content/.add={\bfseries \ttfamily \small}{}
            },
        },
    },
    format userId/.style={
        columns/userId/.style={
            % column name={},
            string type,
            postproc cell content/.code={}
        },
    },
    format movieId/.style={
        columns/movieId/.style={
            % column name={},
            string type,
            postproc cell content/.code={}
        },
    },
    format tag/.style={
        columns/tag/.style={
            % column name={},
            string type,
            postproc cell content/.code={}
        },
    },
    format manufacturer/.style={
        columns/manufacturer/.style={
            % column name={},
            string type,
            postproc cell content/.code={}
        },
    },
    format Lineage/.style={
        columns/Lineage/.style={
            % column name={},
            string type,
            postproc cell content/.code={}
        },
    },
    format method/.style={
        columns/{Provenance Method}/.style={
            % column name={},
            string type,
            postproc cell content/.code={},
            postproc cell  content/.append style={
            @cell content/.add={\bfseries}{}
            },
        },
    },
    format table/.style={
        columns/{Lin. size}/.style={
            % column name={},
            postproc cell content/.code={}
        },
        every head row/.style={before row=\toprule,after row=\midrule},
        every last row/.style={after row=\bottomrule},
        every odd row/.style={before row={\rowcolor{gray!10}}},
    },
    format stats-table1-exp1/.style={
        columns/Total/.style={
            % column name={},
            postproc cell content/.code={}
        },
        columns/Length-L1/.style={
            column name={L[1]},
            postproc cell content/.code={}
        },
        columns/Length-L2/.style={
            column name={L[2]},
            postproc cell content/.code={}
        },
        columns/Length-L3/.style={
            column name={L[3]},
            postproc cell content/.code={}
        },
        columns/Recall-L1/.style={
            column name={L[1]},
            % postproc cell content/.code={}
        },
        columns/Recall-L3/.style={
            column name={L[3]},
            % postproc cell content/.code={}
        },
        every head row/.style={before row=\toprule & & \multicolumn{1}{c}{\textbf{Target Tuple}} & \multicolumn{4}{c}{\textbf{Lineage Size}} & \multicolumn{4}{c}{\textbf{Precision}} & \multicolumn{2}{c}{\textbf{Recall}}\\
        \cmidrule(lr){3-3} \cmidrule(lr){4-7} \cmidrule(lr){8-11} \cmidrule(lr){12-13}\\, after row=\midrule},
        every last row/.style={after row=\bottomrule},
        every odd row/.style={before row={\rowcolor{gray!10}}},
    },
    format stats-table2-exp1/.style={
        columns/Total/.style={
            % column name={},
            postproc cell content/.code={}
        },
        columns/Length-L1/.style={
            column name={L[1]},
            postproc cell content/.code={}
        },
        columns/Length-L2/.style={
            column name={L[2]},
            postproc cell content/.code={}
        },
        columns/Length-L3/.style={
            column name={L[3]},
            postproc cell content/.code={}
        },
        columns/Recall-L1/.style={
            column name={L[1]},
            % postproc cell content/.code={}
        },
        columns/Recall-L2/.style={
            column name={L[2]},
            % postproc cell content/.code={}
        },
        every head row/.style={before row=\toprule & & \multicolumn{1}{c}{\textbf{Target Tuple}} & \multicolumn{4}{c}{\textbf{Lineage Size}} & \multicolumn{4}{c}{\textbf{Precision}} & \multicolumn{2}{c}{\textbf{Recall}}\\
        \cmidrule(lr){3-3} \cmidrule(lr){4-7} \cmidrule(lr){8-11} \cmidrule(lr){12-13}\\, after row=\midrule},
        every last row/.style={after row=\bottomrule},
        every odd row/.style={before row={\rowcolor{gray!10}}},
    },
    format analyst/.style={
        columns/tuple-id/.style={
            column name={Tuple ID},
            string type,
            postproc cell content/.code={}
        },
        columns/related-table/.style={
            column name={Origin Table},
            string type,
            postproc cell content/.code={},
            postproc cell  content/.append style={
            @cell content/.add={\ttfamily}{}
            },
        },
        columns/lineage-level/.style={
            column name={Lineage Level(s)},
            string type,
            postproc cell content/.code={}
        },
        every head row/.style={before row=\toprule & \multicolumn{3}{c}{\textbf{An Analyst's View}} & \multicolumn{2}{c}{\textbf{Hindsight}}\\
        \cmidrule(lr){2-4} \cmidrule(lr){5-6}\\, after row=\midrule},
        every last row/.style={after row=\bottomrule},
        every odd row/.style={before row={\rowcolor{gray!10}}},
    },
    format samples-table/.style={
        every head row/.style={before row=\toprule,after row=\midrule},
        every last row/.style={after row=\bottomrule},
        every odd row/.style={before row={\rowcolor{gray!10}}},
    },
    format samples-nutrients/.style={
        columns/ndb-no/.style={
            column name={\texttt{ndb\_no}},
            string type,
            postproc cell content/.code={}
        },
        columns/derivation-code/.style={
            column name={\texttt{derivation\_code}},
            string type,
            postproc cell content/.code={}
        },
        columns/nutrient-code/.style={
            column name={\texttt{nutrient\_code}},
            string type,
            postproc cell content/.code={}
        },
        columns/nutrient-name/.style={
            column name={\texttt{nutrient\_name}},
            string type,
            postproc cell content/.code={}
        },
        columns/out-val/.style={
            column name={\texttt{out\_val}},
            string type,
            postproc cell content/.code={}
        },
        columns/out-uom/.style={
            column name={\texttt{out\_uom}},
            string type,
            postproc cell content/.code={}
        },
    },
    format stats-table1-exp2/.style={
        columns/name/.style={
            string type,
            postproc cell content/.code={}
        },
        columns/Total/.style={
            % column name={},
            postproc cell content/.code={}
        },
        columns/Length-L1/.style={
            column name={L[1]},
            postproc cell content/.code={}
        },
        columns/Length-L2/.style={
            column name={L[2]},
            postproc cell content/.code={}
        },
        columns/Length-L3/.style={
            column name={L[3]},
            postproc cell content/.code={}
        },
        columns/Length-L4/.style={
            column name={L[4]},
            postproc cell content/.code={}
        },
        columns/Recall-L1/.style={
            column name={L[1]},
            % postproc cell content/.code={}
        },
        columns/Recall-L3/.style={
            column name={L[3]},
            % postproc cell content/.code={}
        },
        columns/Recall-L4/.style={
            column name={L[4]},
            % postproc cell content/.code={}
        },
        every head row/.style={before row=\toprule & & \multicolumn{2}{c}{\textbf{Target Tuple}} & \multicolumn{5}{c}{\textbf{Lineage Size}} & \multicolumn{4}{c}{\textbf{Precision}} & \multicolumn{3}{c}{\textbf{Recall}}\\
        \cmidrule(lr){3-4} \cmidrule(lr){5-9} \cmidrule(lr){10-13} \cmidrule(lr){14-16}\\, after row=\midrule},
        every last row/.style={after row=\bottomrule},
        every odd row/.style={before row={\rowcolor{gray!10}}},
    },
    format stats-table2-exp2/.style={
        columns/name/.style={
            string type,
            postproc cell content/.code={}
        },
        columns/Total/.style={
            % column name={},
            postproc cell content/.code={}
        },
        columns/Length-L1/.style={
            column name={L[1]},
            postproc cell content/.code={}
        },
        columns/Length-L2/.style={
            column name={L[2]},
            postproc cell content/.code={}
        },
        columns/Length-L3/.style={
            column name={L[3]},
            postproc cell content/.code={}
        },
        columns/Length-L4/.style={
            column name={L[4]},
            postproc cell content/.code={}
        },
        columns/Recall-L1/.style={
            column name={L[1]},
            % postproc cell content/.code={}
        },
        columns/Recall-L2/.style={
            column name={L[2]},
            % postproc cell content/.code={}
        },
        columns/Recall-L3/.style={
            column name={L[3]},
            % postproc cell content/.code={}
        },
        every head row/.style={before row=\toprule & & \multicolumn{2}{c}{\textbf{Target Tuple}} & \multicolumn{5}{c}{\textbf{Lineage Size}} & \multicolumn{4}{c}{\textbf{Precision}} & \multicolumn{3}{c}{\textbf{Recall}}\\
        \cmidrule(lr){3-4} \cmidrule(lr){5-9} \cmidrule(lr){10-13} \cmidrule(lr){14-16}\\, after row=\midrule},
        every last row/.style={after row=\bottomrule},
        every odd row/.style={before row={\rowcolor{gray!10}}},
    },
    format stats-table1-exp3/.style={
        columns/name/.style={
            string type,
            postproc cell content/.code={}
        },
        columns/Total/.style={
            % column name={},
            postproc cell content/.code={}
        },
        columns/Length-L1/.style={
            column name={L[1]},
            postproc cell content/.code={}
        },
        columns/Length-L2/.style={
            column name={L[2]},
            postproc cell content/.code={}
        },
        columns/Length-L3/.style={
            column name={L[3]},
            postproc cell content/.code={}
        },
        columns/Recall-L1/.style={
            column name={L[1]},
            % postproc cell content/.code={}
        },
        columns/Recall-L2/.style={
            column name={L[2]},
            % postproc cell content/.code={}
        },
        columns/Recall-L3/.style={
            column name={L[3]},
            % postproc cell content/.code={}
        },
        every head row/.style={before row=\toprule & & \multicolumn{2}{c}{\textbf{Target Tuple}} & \multicolumn{4}{c}{\textbf{Lineage Size}} & \multicolumn{4}{c}{\textbf{Precision}} & \multicolumn{3}{c}{\textbf{Recall}}\\
        \cmidrule(lr){3-4} \cmidrule(lr){5-8} \cmidrule(lr){9-12} \cmidrule(lr){13-15}\\, after row=\midrule},
        every last row/.style={after row=\bottomrule},
        every odd row/.style={before row={\rowcolor{gray!10}}},
    },
    format stats-table2-exp3/.style={
        columns/name/.style={
            string type,
            postproc cell content/.code={}
        },
        columns/Total/.style={
            % column name={},
            postproc cell content/.code={}
        },
        columns/Length-L1/.style={
            column name={L[1]},
            postproc cell content/.code={}
        },
        columns/Length-L2/.style={
            column name={L[2]},
            postproc cell content/.code={}
        },
        columns/Length-L3/.style={
            column name={L[3]},
            postproc cell content/.code={}
        },
        columns/Recall-L1/.style={
            column name={L[1]},
            % postproc cell content/.code={}
        },
        columns/Recall-L2/.style={
            column name={L[2]},
            % postproc cell content/.code={}
        },
        every head row/.style={before row=\toprule & & \multicolumn{2}{c}{\textbf{Target Tuple}} & \multicolumn{4}{c}{\textbf{Lineage Size}} & \multicolumn{4}{c}{\textbf{Precision}} & \multicolumn{2}{c}{\textbf{Recall}}\\
        \cmidrule(lr){3-4} \cmidrule(lr){5-8} \cmidrule(lr){9-12} \cmidrule(lr){13-14}\\, after row=\midrule},
        every last row/.style={after row=\bottomrule},
        every odd row/.style={before row={\rowcolor{gray!10}}},
    }
}

\newcommand{\notimplies}{\centernot\implies}
\newcommand{\floor}[1]{\left\lfloor #1 \right\rfloor}

% \theoremstyle{definition}
\newtheorem{runexample}{Running Example}
\newtheorem{example-withrun}{Example}[runexample]

% \theoremstyle{definition}
\newtheorem{runexperiment}{Running Experiment}
\newtheorem{experiment-withrun}{Experiment}[runexperiment]

\newcommand{\fncolor}[1]{\textcolor{YellowOrange}{#1}}
\newcommand{\paramcolor}[1]{\textcolor{JungleGreen}{#1}}
\newcommand{\cmtcolor}[1]{\textcolor{Gray}{#1}}

\makeatletter
\newcommand{\removelatexerror}{\let\@latex@error\@gobble}
\makeatother
