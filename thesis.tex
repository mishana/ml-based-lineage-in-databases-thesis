%
% You may wish to use some of the following options of the iitthesis
% package:
%
% fullpageDraft      avoid the margins necessary for proper binding and
%   just view or print a draft.
% beforeDefense      makes the personal acknowledgements invisible;
%   use this to print the copies you submit initially to the grad school
%   for sending to the opponent panel, i.e. thesis readers (who shouldn't
%   see those parts). For the final submission, after having successfully
%   defended - drop this option.
% noabbrevs          no notation & abbreviations list will be included
%   in the thesis.
%
% Additionally, you must specify the degree for which you're writing
% your thesis (MSc/PhD/MArch etc.)
%
\documentclass[MSc,noabbrevs,beforeDefense]{misc/iitthesis}


% Definitions of info fields for the thesis - subject, advisor,
% faculty, acknowledgements, etc. etc. The thesis-fields file 
% contains Hebrew text, and should use the UTF-8 character set
% encoding (not iso-8859-8-i or windows codepage 1255).
% This file contains definitions of various fields used
% in various places throughout the thesis (in the title
% pages mostly). Whatever isn't define here has some
% default (and usually irrelevant) text.

\authorEnglish{Michael Leybovich}
\authorHebrew{מיכאל לייבוביץ׳}

\titleEnglish{ML Based Lineage in Databases}
\titleHebrew{למידת מכונה של מוצאות רשומות במסדי נתונים}

\disciplineEnglish{Computer Science}
\disciplineHebrew{מדעי המחשב}

\supervisionEnglish{This research was carried out under the supervision of Prof. Oded Shmueli, in the Faculty of Computer Science.}
\supervisionHebrew{המחקר בוצע בהנחייתו של פרופסור עודד שמואלי, בפקולטה למדעי המחשב.}

\GregorianDateEnglish{April 2021}
\GregorianDateHebrew{אפריל \textenglish{2021}}
\JewishDateEnglish{Nisan 5781}
\JewishDateHebrew{ניסן התשפ"א}

%\financialAcknowledgementEnglish{The Technion's funding of this research is hereby acknowledged.}
%\financialAcknowledgementHebrew{הכרת תודה מסורה לטכניון על מימון מחקר זה.}

\publicationInfoEnglish{%
% The following may not be true regarding your own thesis...
% (The grad school guidelines now require that you mention the following regarding publications of your thesis work; but of course, remove this parenthesized note...; this is to be found in the \texttt{thesis-fields.tex} file. Note also that the document may need to be processed several times before the list of publications actually appears)

Some results in this thesis have been published in an article by the author (Michael Leybovich) and the advisor (Oded Shmueli) during the course of this master research, the most up-to-date versions of which is:
 
% No need to specifically non-cite the items, all of the bib file's contents
% will appear here regardless
% \nociteacks{firstwork-foracks}
% \nociteacks{secondwork-foracks}
\butcheredbibliography{pubinfo}{front/pubinfo}
}

\publicationInfoHebrew{%

חלק מן התוצאות בחיבור זה פורסמו כמאמרים מאת המחבר (מיכאל לייבוביץ׳) והמנחה (עודד שמואלי) במהלך המחקר שלהמחבר, אשר גרסאותיהם העדכניות ביותר הינן:%

\begin{otherlanguage}{english}%
% No need to mention the bibliography file this time, as it has already been used in
% the English invocation
\butcheredbibliography{pubinfo}{front/pubinfo}
\end{otherlanguage}%
}

\thesisbibfiles{back/general}
\thesisbibstyle{alpha}


% Personal acknowledgements (are only used for the post-exam
% version)
\include{front/personal-acks}

% A separate file for the abstract - in English and in Hebrew, so
% you must make sure it's also in the UTF-8 character set encoding.
%
% This file contains the abstract part of your thesis - in English and
% in Hebrew (within \abstractEnglish and \abstractHebrew respectively).
%
% Notes:
% - This file uses the UTF-8 character set encoding for the Hebrew
%   text not to get garbled. Keep it that way.
% - Assuming your thesis is mainly in English, Graduate School 
%   regulations mandate the following lengths for the abstracts:
%
%      Language    Min. Length   Max. Length
%     ---------------------------------------
%      English       200 words     500 words
%      Hebrew      1,000 words   2,000 words
%
%   so that the Hebrew abstract typically has some content from
%   the English introduction and an overview of the results, not
%   present in the English; it is not just a translation.

\abstractEnglish{

There has been extensive research on data provenance. Previous works were concerned with annotating the results of database (DB) queries with provenance information which is useful in explaining query results at various resolution levels. In this work, we track the lineage of tuples throughout their database lifetime. That is, we consider a scenario in which tuples (records) that are produced by a query may affect other tuple insertions into the DB, as part of a normal workflow. As time goes on, provenance explanations for such tuples become deeply nested, increasingly consuming  space, and resulting in decreased clarity and readability.
\par We present a novel approach for approximating lineage tracking. We use Machine Learning (ML) and Natural Language Processing (NLP) techniques; mainly, word embedding. 
The basic idea is summarizing (and approximating) the lineage of each tuple via a small set of constant-size vectors (the number of vectors per-tuple is a hyperparameter). For explicitly (and independently of DB contents)  inserted tuples -  the vectors are obtained via a pre-trained word vectors model over their underlying database domain ``text". During the execution of a query, we construct the lineage vectors of the final (and intermediate) result tuples in a similar fashion to that of semiring-based exact provenance calculations. We extend the $+$ and $\cdot$ operations to generate sets of lineage vectors, while emphasizing the ability to propagate information and preserve the compact representation. Therefore, our solution does not suffer from space complexity blow-up over time. Another significant benefit of our approach is a ``natural ranking" of explanations to the existence of a tuple in the DB.
\par We devise a genetics-inspired improvement to our basic method. The data columns of an entity (and potentially other columns) are a tuple’s basic properties, i.e., the ``genes" that combine to form its genetic code. In this setting, finding the lineage of a tuple in the DB is analogous to finding its predecessors via DNA examination. We design an alternative lineage tracking mechanism, that of keeping track of and querying lineage (via embeddings) at the column (``gene") level. This way, we manage to better distinguish between the provenance features and the textual characteristics of a tuple.
\par We further introduce several improvements and extensions to the basic method:
\begin{itemize}
    \item Emphasizing important columns with a query-dependent column weighting.
    \item Filtering non-contributing tuples using Bloom Filters of queries.
    \item Filtering non-contributing tuples by a tuple creation timestamp.
    \item Similarity weighting with query dependency DAG.
    \item Extending the method to track where provenance.
\end{itemize}
\par We integrate our lineage computations into the PostgreSQL system via an extension (ProvSQL) and experimentally exhibit useful results in terms of accuracy against exact, semiring-based, justifications.  In the experiments, we focus on tuples with multiple generations of tuples in their lifelong lineage and analyze them in terms of direct and distant lineage. The examples we present suggest a high usefulness potential for the proposed approximate lineage methods and the further suggested enhancements. This especially holds for the column-based vectors method which exhibits high precision and per-level recall.
\par Finally, to practically realize our methods, we need to search efficiently for a set of vectors that maximizes set-wise similarity. We outline such a search algorithm, which is based on encoding a set of vectors via a ``long" single vector. 


} % end of English abstract


\abstractHebrew{

% Note that certain commands don't work that well in Hebrew "mode".
% If this happens to you, try wrapping the command within a
% \textenglish{ } - that may (or may not) help.

יוחסין של נתונים הוא נושא שנחקר רבות בשנים האחרונות.
עבודות קודמות עסקו בתיוג של תוצאות שאילתות על גבי מסדי נתונים על ידי ׳׳מטא נתונים״ המעידים על היוחסין של התוצאות.
מטא נתונים אלה הוכחו כשימושיים בהסבר של תוצאות שאילתות ברמות פירוט שונות.

במחקר זה אנו שואפים להתחקות אחר המוצאות של רשומות במסדי נתונים לאורך חייהן, תוך הבחנה בין רשומות שהוכנסו באופן ידני ובלתי תלוי בתכולת מסד הנתונים (אלו ״אבני הבנין״ שלנו) לבין רשומות הנוצרות על ידי שאילתא (או, באופן כללי, אשר תלויות בתוכן מסד הנתונים).
כלומר, אנחנו מניחים תסריט שבו רשומות הנוצרות על ידי שאילתא עלולות להשפיע על הכנסתן של רשומות אחרות למסד הנתונים, וזאת כחלק מתזרים עבודה תקני של המערכת.
בתרחיש אמיתי - רשומות שתלויות בתכולת מסד הנתונים יווצרו באחת מהדרכים הבאות:
\begin{itemize}
    \item היררכיה של  
    מבטים מתוחזקים -
    כך שכל 
    מבט
    יכול להיות תלוי הן בתכולת מסד הנתונים והן
    במבטים
    שהוגדרו בשלב מוקדם יותר.
    \item רשומות אשר מוכנסות למסד הנתונים באמצעות פקודת 
    \textenglish{\texttt{INSERT INTO SELECT}}.
    \item רשומות אשר מוכנסות למסד הנתונים באמצעות פקודת 
    \textenglish{\texttt{UPDATE}}.
    \item תוצאת שאילתא עם שדות שהוספו באופן ידני, אשר מוכנסת בחזרה לטבלה אחרת במסד הנתונים.
\end{itemize}

עם חלוף הזמן, אילנות היוחסין (וההסברים הנובעים מהם) של רשומות אלה הופכים מקוננים ומסובכים מאוד מבחינת צריכת זיכרון, זמן חישוב עבור מוצאות רשומות, בהירות וקריאות. 
אנו מציגים גישה חדשה לחישוב מקורב של מוצאות רשומות תוך שימוש בטכניקות של למידת מכונה ועיבוד שפה טבעית (ובעיקר, שיכון מילים במרחב אוקלידי).

הרעיון המרכזי הוא לסכם (באופן מקורב) את היוחסין של כל רשומה על ידי קבוצה סופית של וקטורים (מספר הוקטורים פר-רשומה הוא היפרפרמטר של המערכת).
עבור רשומות המוכנסות למסד הנתונים באופן ידני - הוקטורים מחושבים באמצעות מודל מאומן מראש של וקטורי מילה על גבי ה״טקסט״ של הרשומות.
בזמן תהליך ביצוע של שאילתא על גבי מסד הנתונים, אנו מחשבים את וקטורי היוחסין של הרשומות החדשות בתוצאה של השאילתא באופן דומה לחישוב מדוייק המבוסס חצאי-חוגים (מבנה אלגברי).
אנו מכלילים את פעולות החיבור והכפל לעבודה על קבוצות של וקטורי יוחסין, תוך שימת דגש על היכולת לפעפע את מידע היוחסין ושמירה על הייצוג הקומפקטי של מספר סופי וחסום של וקטורי יוחסין פר-רשומה.
\newpage
כתוצאה, אנחנו מקבלים קבוצות של וקטורי יוחסין עבור רשומות בתוצאה של שאילתא. רשומות חדשות אלה (והמידע בעניין מוצאותיהן) עשויות להתווסף למסד הנתונים בשלב מאוחר יותר (כתלות ביישום הספציפי).

אנו מציגים שיפור משמעותי לשיטה הבסיסית (שפיתחנו) בהשראת אנאלוגיה לתחום הגנטי.
העמודות של יישות (ופוטנציאלית עמודות נוספות) הן המאפיינים הבסיסיים של רשומה באותה יישות, כלומר, הגנים שמרכיבים את הקוד הגנטי שלה.
תחת הנחות אלה, מציאת המוצאות של רשומה במסד הנתונים שקולה למציאת האבות הקדמונים שלה באמצעות בדיקות דנ״א.
אם כך, אנו מציעים לתחזק ולחפש מוצאות של רשומות באמצעות וקטורי יוחסין ברמת העמודות (גנים) ולא ברמת רשומות שלמות. בצורה זו, אנו מצליחים להבחין בצורה טובה יותר בין תכונות היוחסין לבין המאפיינים הטקסטואליים של רשומה.

לאחר תיאור הרעיון המרכזי, אנו מציגים שיפורים והרחבות לשיטה, שנועדו להפוך אותה לפרקטית ושימושית עבור אנאליסט. אותו אנאליסט, שירצה להשתמש בוקטורי היוחסין על מנת למצוא את מוצאותיה של רשומה ספציפית במסד הנתונים, יוכל לעשות זאת על ידי השוואה בין וקטורי היוחסין של הרשומה לוקטורי היוחסין של קבוצת עניין במסד הנתונים (כל קבוצה של רשומות), ולדרג אותן לפי ״קירבה״ - באופן דומה למתודולוגיה של חיפוש מילים קרובות סמנטית באמצעות השוואת וקטורי מילה.

אנו משלבים את החישוב המקורב של מוצאות רשומות במערכת לניהול מסדי נתונים
\textenglish{PostgreSQL}
באמצעות ההרחבה
\textenglish{ProvSQL}
ומציגים תוצאות בעלות תועלת מבחינת דיוק בהשוואה לחישוב מדויק המבוסס חצאי-חוגים.
אנו טוענים כי הפיתרון המוצע אינו סובל מניפוח יתר בסיבוכיות מקום לאורך זמן,
ולכן הופך מערכת לחקר יוחסין של רשומות ברת ביצוע תחת האילוצים שתוארו לעיל.
יתרון משמעותי של השיטה שלנו הוא ״דירוג טבעי״ של מוצאות רשומות.

לסיום, כדי לתמוך במימוש יעיל, אנו זקוקים לשיטה של חיפוש קבוצה של וקטורים הקרובה ביותר לקבוצה נתונה. 
אנו מציגים שיטת חיפוש כזו המבוססת על הרעיון של ייצוג של קבוצת וקטורים באמצעות וקטור ״ארוך״ בודד.

% בית הספר ללימודי מוסמכים מנחה מספר הנחיות לגבי התקציר בעברית:
% \begin{itemize}
% \item על התקציר להיכתב במשפטים מקושרים שלמים.
% \item בדרך-כלל אין לציין בתקציר מקורות ספרותיים וציטוטים.
% \item אין להתייחס למספר של פרק, סעיף, נוסחה, ציור או טבלה שבגוף החיבור, ואין להשתמש בקיצורים, סמלים ומונחים לא מקובלים, אלא אם יש בתקציר די מקום לזיהויים.
% \end{itemize}



% \subsection*{\texthebrew{תת-חלק בתקציר המורחב}}

% תוכן מקוצר לגבי נושא מסוים. התייחסות ל\emph{מושג} מסוים שהחיבור בוחן. וכולי וכולי.


% \subsection*{\texthebrew{נקודה מעניינת לגבי העמודים בעברית}}

% שימו לב כי העמודים בעברית אמורים להיות מיוצרים בסדר ה''הפוך'', הווה אומר העמוד האחרון בקובץ ה-\textenglish{PDF} הוא הכריכה העברית, לפניו השער העברי, ודפי התקציר צריכים להופיע בסדר הפוך (וכן במספור רומי, לפי נהלי הטכניון). כך אם נתבונן במספר שבתחתית עמוד זה \textenglish{---} אשר צריך להיות העמוד הראשון בתקציר-המורחב מבחינת רצף התוכן, והינו העמוד האחרון מבין עמודי התקציר-המורחב אחרון בקובץ ה-\textenglish{PDF} \textenglish{---} נמצא את המספר \textenglish{i} ...

% \newpage

% ... ואילו עמוד זה של התקציר-המורחב בעברית \textenglish{---} שהינו העמוד השני בתקציר-המורחב מבחינת רצף התוכן, ונמצא ראשון בקובץ ה-\textenglish{PDF} \textenglish{---} ממוספר ב-\textenglish{ii}. המטרה במספור בסדר ה"הפוך" היא, שבעת ההדפסה לא יהיה צורך להפוך דפים, לשנות את סדרם וכולי \textenglish{---} רק להדפיס ולכרוך.

} % end of Hebrew abstract


% Just write down your abbervations here - or comment-out the command

% \abbreviationsAndNotation{
% \begin{tabular}{p{2cm}@{:\quad}l}
% QED & quod erat demonstrandum (``what was to be shown'')\\
% $c$ & the speed of light \\
% $a\pm b$ & the closed interval $\left[a-b,a+b\right]$ \\
% \end{tabular}
% }


% Additional machinery relevant to any thesis
% (it's not part of the document class because it's not absolutely
% necessary and not everyone might like it)
\usepackage{misc/iitthesis-extra}

% Definitions useful for anything you write, which you also
% include in any articles, presentations, HW assignments and other
% documents. May contains macros for notation algebra, logic,
% calculus and other fields, as well as general shorthands and
% LaTeX tricks, and package use commands
% General-purpose definitions and inclusions
% you are using in any document 
% (regardless of its class and style files used),
% e.g. package uses:

%\usepackage{xspace}

% and macros/command defintions:

%\newcommand{\complexityclass}[1]{{\bf #1}\xspace}
%\newcommand{\NPTIME}{\complexityclass{NP}}

% For this template, we'll only have one single command,
% necessary for including graphics...
\usepackage[table, dvipsnames]{xcolor}
\usepackage{graphicx}% http://ctan.org/pkg/graphicx
\usepackage{tikz}
\usepackage[algo2e, linesnumbered, boxed, vlined]{algorithm2e}
\usepackage{caption}

% Definitions, settings and tweaks for this thesis specifically
% This file should contain your own definitions specific
% only to this thesis;

% What it contains below is used 
% simply for generating dummy text in the sample 
% content provided with the template (see mainchap1.tex);
% so you can safely delete this when creating your own
% thesis

\usepackage{centernot}
\usepackage{adjustbox}
\usepackage{float}
\usepackage{pgfplotstable}
\pgfplotstableset{
    /color cells/min/.initial=0,
    /color cells/max/.initial=1000,
    /color cells/textcolor/.initial=,
    %
    % Usage: 'color cells={min=<value which is mapped to lowest color>, 
    %   max = <value which is mapped to largest>}
    color cells/.code={%
        \pgfqkeys{/color cells}{#1}%
        \pgfkeysalso{%
            postproc cell content/.code={%
                %
                \begingroup
                %
                % acquire the value before any number printer changed
                % it:
                \pgfkeysgetvalue{/pgfplots/table/@preprocessed
                  cell content}\value
                \ifx\value\empty
                    \endgroup
                \else
                \pgfmathfloatparsenumber{\value}%
                \pgfmathfloattofixed{\pgfmathresult}%
                \let\value=\pgfmathresult
                %
                % map that value:
                \pgfplotscolormapaccess
                    [\pgfkeysvalueof{/color cells/min}:\pgfkeysvalueof{/color
                      cells/max}]
                    {\value}
                    {\pgfkeysvalueof{/pgfplots/colormap name}}%
                % now, \pgfmathresult contains {<R>,<G>,<B>}
                % 
                % acquire the value AFTER any preprocessor or
                % typesetter (like number printer) worked on it:
                \pgfkeysgetvalue{/pgfplots/table/@cell content}\typesetvalue
                \pgfkeysgetvalue{/color cells/textcolor}\textcolorvalue
                %
                % tex-expansion control
                % see http://tex.stackexchange.com/questions/12668/
                \toks0=\expandafter{\typesetvalue}%
                \xdef\temp{%
                    \noexpand\pgfkeysalso{%
                        @cell content={%
                            \noexpand\cellcolor[rgb]{\pgfmathresult}%
                            \noexpand\definecolor{mapped
                              color}{rgb}{\pgfmathresult}%
                            \ifx\textcolorvalue\empty
                            \else
                                \noexpand\color{\textcolorvalue}%
                            \fi
                            \the\toks0 %
                        }%
                    }%
                }%
                \endgroup
                \temp
                \fi
            }%
        }%
    },
    format index/.style={
        columns/index/.style={
            column name={},
            string type,
            postproc cell content/.code={},
            postproc cell  content/.append style={
            @cell content/.add={\bfseries \ttfamily \small}{}
            },
        },
    },
    format userId/.style={
        columns/userId/.style={
            % column name={},
            string type,
            postproc cell content/.code={}
        },
    },
    format movieId/.style={
        columns/movieId/.style={
            % column name={},
            string type,
            postproc cell content/.code={}
        },
    },
    format tag/.style={
        columns/tag/.style={
            % column name={},
            string type,
            postproc cell content/.code={}
        },
    },
    format manufacturer/.style={
        columns/manufacturer/.style={
            % column name={},
            string type,
            postproc cell content/.code={}
        },
    },
    format Lineage/.style={
        columns/Lineage/.style={
            % column name={},
            string type,
            postproc cell content/.code={}
        },
    },
    format method/.style={
        columns/{Provenance Method}/.style={
            % column name={},
            string type,
            postproc cell content/.code={},
            postproc cell  content/.append style={
            @cell content/.add={\bfseries}{}
            },
        },
    },
    format table/.style={
        columns/{Lin. size}/.style={
            % column name={},
            postproc cell content/.code={}
        },
        every head row/.style={before row=\toprule,after row=\midrule},
        every last row/.style={after row=\bottomrule},
        every odd row/.style={before row={\rowcolor{gray!10}}},
    },
    format stats-table1/.style={
        columns/Total/.style={
            % column name={},
            postproc cell content/.code={}
        },
        columns/Length-L1/.style={
            column name={L[1]},
            postproc cell content/.code={}
        },
        columns/Length-L2/.style={
            column name={L[2]},
            postproc cell content/.code={}
        },
        columns/Length-L3/.style={
            column name={L[3]},
            postproc cell content/.code={}
        },
        columns/Recall-L1/.style={
            column name={L[1]},
            % postproc cell content/.code={}
        },
        columns/Recall-L3/.style={
            column name={L[3]},
            % postproc cell content/.code={}
        },
        every head row/.style={before row=\toprule & & \multicolumn{1}{c}{\textbf{Target Tuple}} & \multicolumn{4}{c}{\textbf{Lineage Size}} & \multicolumn{4}{c}{\textbf{Precision}} & \multicolumn{2}{c}{\textbf{Recall}}\\
        \cmidrule(lr){3-3} \cmidrule(lr){4-7} \cmidrule(lr){8-11} \cmidrule(lr){12-13}\\, after row=\midrule},
        every last row/.style={after row=\bottomrule},
        every odd row/.style={before row={\rowcolor{gray!10}}},
    },
    format stats-table2/.style={
        columns/Total/.style={
            % column name={},
            postproc cell content/.code={}
        },
        columns/Length-L1/.style={
            column name={L[1]},
            postproc cell content/.code={}
        },
        columns/Length-L2/.style={
            column name={L[2]},
            postproc cell content/.code={}
        },
        columns/Length-L3/.style={
            column name={L[3]},
            postproc cell content/.code={}
        },
        columns/Recall-L1/.style={
            column name={L[1]},
            % postproc cell content/.code={}
        },
        columns/Recall-L2/.style={
            column name={L[2]},
            % postproc cell content/.code={}
        },
        every head row/.style={before row=\toprule & & \multicolumn{1}{c}{\textbf{Target Tuple}} & \multicolumn{4}{c}{\textbf{Lineage Size}} & \multicolumn{4}{c}{\textbf{Precision}} & \multicolumn{2}{c}{\textbf{Recall}}\\
        \cmidrule(lr){3-3} \cmidrule(lr){4-7} \cmidrule(lr){8-11} \cmidrule(lr){12-13}\\, after row=\midrule},
        every last row/.style={after row=\bottomrule},
        every odd row/.style={before row={\rowcolor{gray!10}}},
    },
    format analyst/.style={
        columns/tuple-id/.style={
            column name={Tuple ID},
            string type,
            postproc cell content/.code={}
        },
        columns/related-table/.style={
            column name={Origin Table},
            string type,
            postproc cell content/.code={},
            postproc cell  content/.append style={
            @cell content/.add={\ttfamily}{}
            },
        },
        columns/lineage-level/.style={
            column name={Lineage Level(s)},
            string type,
            postproc cell content/.code={}
        },
        every head row/.style={before row=\toprule & \multicolumn{3}{c}{\textbf{An Analyst's View}} & \multicolumn{2}{c}{\textbf{Hindsight}}\\
        \cmidrule(lr){2-4} \cmidrule(lr){5-6}\\, after row=\midrule},
        every last row/.style={after row=\bottomrule},
        every odd row/.style={before row={\rowcolor{gray!10}}},
    }
}

\newcommand{\notimplies}{\centernot\implies}
\newcommand{\floor}[1]{\left\lfloor #1 \right\rfloor}

% \theoremstyle{definition}
\newtheorem{runexample}{Running Example}
\newtheorem{example-withrun}{Example}[runexample]

% \theoremstyle{definition}
\newtheorem{runexperiment}{Running Experiment}
\newtheorem{experiment-withrun}{Experiment}[runexperiment]

\newcommand{\fncolor}[1]{\textcolor{YellowOrange}{#1}}
\newcommand{\paramcolor}[1]{\textcolor{JungleGreen}{#1}}
\newcommand{\cmtcolor}[1]{\textcolor{Gray}{#1}}

\makeatletter
\newcommand{\removelatexerror}{\let\@latex@error\@gobble}
\makeatother


% If you are using WinEdt, and using a publication list on the the
% acknowledgements page, and are having problems getting your document
% to compile with the 'PDFLaTeXify' button, try uncommenting the
% following two lines;
% Also, you will need to PDFLaTeXify at least twice, as WinEdt misses
% an extra run. See also:
% http://tex.stackexchange.com/q/41727/5640
\usepackage{multibib}
\newcites{pubinfo}{Acknowledgement page references}
\def\iitthesisextramultibibdefs{}

\begin{document}

% Front Matter
% ------------

% The following command will typeset the outer cover page, the
% inner title page, the acknowledgements page, etc. - everything
% up to but not including the introduction
\makefrontmatter

% Main Matter
% ------------
%
% To conform to Technion regulations, the main matter should begin
% with an introduction (including a survey of relevant past work):
%
\chapter{Introduction}
\label{chap:intro}
% do we need to add TOC lines?

%\begin{figure}
%  \centering
%  \includegraphics[width=0.75\textwidth]{main/graphics/a_blowup.pdf}
%  \caption{This is a caption}
%\end{figure}

\subsection*{Data Lineage}
The focus of this work is providing explanations (or justifications) for the existence of tuples in a Database Management System (DBMS). These explanations are also known as \textit{data provenance} \cite{cheney2009provenance}.
Provenance in the literature \cite{Ives2008, Deutch2015} often refers to forms of ``justifying'' query results. That is, the provenance context is the state of the database (DB) just before the query execution. The specific type of provenance on which we focus is \textit{lineage} \cite{Cui:2000:TLV:357775.357777}, namely a collection of DB tuples whose existence led to the existence of tuple \textit{t} in a query result. 


\subsection*{Distant Lineage}
We take a more comprehensive look. We track lineage of tuples throughout their existence, while distinguishing between tuples that are inserted explicitly and independently of DB content (these are our ``building blocks'') and tuples that are inserted via a query (or, more generally, that depend on the contents of the DB).
In a real-life setting - tuples that are inserted via a query can be one of the following:
\begin{itemize}
    \item A hierarchy of materialized views - where each view can depend both on the DB and on previously defined views.
    \item Tuples that are inserted via a SQL \texttt{INSERT INTO SELECT} statement.
    \item Tuples that are inserted via a SQL \texttt{UPDATE} statement.
    \item A query result with explicitly added data fields, that is added back to another table in the DB. For example, get names of customers retrieved from an \texttt{orders} table, calculate some non-database-resident ``psychological profile" for each customer and insert both to a \texttt{customer\_profile} table for future recommendations.
\end{itemize}


\subsection*{Approximate Lineage}
As time goes on, provenance information for tuples that are inserted via a query may become complex (e.g., by tracking semiring formulas, as presented in \cite{green2007provenance}, or circuits as presented  in \cite{Deutch2014, Senellart2017}). Thus, the goal of this work is providing ``simple to work with'' and useful approximate lineage (using ML and NLP techniques), while requiring only a constant additional space per tuple. This approximate lineage is compared against state of the art ``exact provenance tracking system'' in terms of explainability and maintainability.\\


\subsection*{Main Contributions}
\par The main contributions of this work are as follows:
\begin{enumerate}
    \item The efficient, with constant additional space per tuple, usage of word vectors to encode \textit{lifelong} lineage.
    \item A family of algorithms and enhancements that render the approach practical for both direct (i.e., current DB state) and indirect (i.e., all history) lineage computations.
    \item Thorough experimentation which exhibits high usefulness potential.
\end{enumerate}


\subsection*{Organization}
The rest of this work is organized as follows.
Previous work on provenance is surveyed in depth in Chapter 2.
In Chapter 3 we discuss relational embeddings and the motivation to use them for lineage encoding.
In Chapter 4 we introduce basic algorithms for creating per-tuple lineage encoding vectors.
In Chapter 5 we adapt these basic algorithms to per-column lineage encoding vectors.
Chapter 6 discusses improvements to the previously presented algorithms.
In Chapter 7 we present thorough experimental results.
In Chapter 8 we provide further discussion on a few topics presented in this paper.
We conclude in Chapter 9, in which we also outline directions for future research.

% \subsection*{An unnumbered subsection}

% You may want to break up the intro into parts with titles. Subsectioning without numbering is an option you might want to consider.

% Some people include a specific section overviewing the results ("In Chapter so-and-so, we will see how etc.") which is also a way of describing the structure of the thesis. But this is not necessary.

% \subsection*{Thesis options and appearance}

% Please note that the \texttt{iitthesis} class has several options when you use it, such as:
% \begin{itemize}
% \item \texttt{fullpageDraft} to avoid the margins necessary for proper binding when you make the final print
% \item \texttt{beforeDefense} makes the personal acknowledgements invisible; use this to print the copies you submit initially to the grad school for sending to the opponent panel, i.e. thesis readers (who shouldn't see those parts). For the final submission, after having successfully defended --- drop this option. 
% \item \texttt{noabbrevs} no notation \& abbreviations list will be included in the thesis.
% \end{itemize}

% \subsection*{Hebrew font}

% The \texttt{iitthesis} document class uses the David CLM font family for Hebrew text. CLM is a shorthand for ``Culmus'' (\texthebrew{קולמוס}) --- the name of a freely-available Hebrew font package. It may be bundled with your LaTeX distribution, or otherwise, must be available as system fonts. If you're missing the Culmus fonts, try adding an appropriate package from your LaTeX distribution or system distribution; alternatively, you might want to visit the Culmus project page at \url{http://culmus.sourceforge.net/} and download and install the fonts manually.

% \subsection*{Setting thesis meta-data and publication information}

% The document class used to generate this document defines several commands you can use to set information  regarding your thesis, which is used in the title pages and elsewhere in the front matter.  Every (or almost every) command has an English and a Hebrew variant, with a \texttt{English} or \texttt{Hebrew} suffix to the command name. Examples:
% \begin{itemize}
% \item \verb|\titleHebrew|, \verb|\titleEnglish|
% \item \verb|\authorHebrew|, \verb|\authorEnglish|
% \item \verb|\JewishDateHebrew|, \verb|\JewishDateEnglish|
% \item \verb|\GregorianDateHebrew|, \verb|\GregorianDateEnglish|
% \item \verb|\publicationinfoHebrew|, \verb|\publicationinfoEnglish|
% \end{itemize}

% The file \texttt{misc/thesis-fields.tex} contains invocations of several such commands (some of them commented-out with \texttt{\%}), and some additional information about them.



%
% and then cover:
% - The methods used in the research
% - The research results
% - Discussion and conclusions from the results
%
% but not necessarily with a specific chapter for each of them.
%
% Then you have your main chapters (although these might still
% include an initial chapter on technical preliminaries, experimental
% system setup, and/or a final chapter with summary, discussion and further
% research direction or questions)

\chapter{Classic Provenance}
\label{chap:classic_provenance}

\section{What is Provenance?}

\subsection{Definition}
% Dictionary entry
Provenance - \textit{source, origin} \cite{prov_dict}. 
% General explanation
In computing, provenance information describes the origins and the history of data within its lifetime. When talking about database management systems, the commonly used term is \textit{data provenance} \cite{cheney2009provenance}. The idea behind data provenance is keeping additional information (meta-data) allowing us to easily answer a large number of ``meta-questions".
% Provenance as explanations for query results
\par Data provenance helps with providing explanations for the existence of tuples in a query result. The context of these explanations is the DB state prior to the query execution. \\

% Intro to notions of data provenance
\subsection{Related Work} Over the past 15 years, provenance research has advanced in addressing both theoretical \cite{cheney2009provenance, green2007provenance, Deutch2014} 
and practical \cite{Ives2008, Karvounarakis2010, Deutch2015, Deutch2017, Senellart2018, DBLP:conf/cidr/IvesZHZ19} 
aspects. 
In particular, several different notions of data provenance (\textit{lineage}, \textit{why}, \textit{how} and \textit{where}) were formally defined \cite{Cui:2000:TLV:357775.357777, DBLP:conf/icdt/BunemanKT01,  cheney2009provenance}.
% Related Work - Provenance Approximation and Summarization
\par A few prior works \cite{approx_lineage, approx_PROX, approx_summary, approx_why_and_why_not} focus on approximate (or summarized) provenance. That is, seeking a compact representation of the provenance at the possible cost of information loss in an attempt to deal with the growing size and complexity of exact provenance information in real-life systems.


\section{Provenance Semirings}\label{sec:semiring provenance}
% Overview and definition - Val Tannen and T.J. Green
\subsection{Overview}\footnotemark
\footnotetext{Portions of this section were adapted from Green et al. \cite{green2007provenance} and Karvounarakis et al. \cite{Karvounarakis:2012:SDQ:2380776.2380778}.}
\textit{Provenance semirings} have been introduced by Green et al. \cite{green2007provenance} as a formalism for data provenance. These semirings have been shown \cite{Karvounarakis:2012:SDQ:2380776.2380778} to generalize previous works such as lineage \cite{Cui:2000:TLV:357775.357777} and why-provenance \cite{DBLP:conf/icdt/BunemanKT01}. The main idea is based on annotating every tuple in the DB with elements from some algebraic structure $(K,+,\cdot,0,1)$\footnotemark
\footnotetext{$K$ is a set, containing two distinguished elements $0,1$; and $+,\cdot$ are binary operators on elements from $K$.}
resulting in \textit{K-relations}\footnotemark, 
\footnotetext{Informally, relations in which tuples are annotated with elements from $K$.}
and extending $\mathcal{RA^+}$ (relational algebra, excluding the difference operator) to operate on $K$-relations via definitions in terms of the abstract + and $\cdot$ operations of $K$:
\begin{itemize}
    \item + corresponds to alternative use of data (union - $\cup$ or projection - $\Pi$);
    \item $\cdot$ corresponds to joint use of data (join - $\bowtie$);
    \item $0$ and $1$ are used for selection ($\sigma$) predicates.
\end{itemize}
Note, this method has to comply with common $\mathcal{RA^+}$ identities in order to correctly extend to $K$-relations:
\begin{itemize}
    \item join and union are both associative and commutative;
    \item union has identity $\emptyset$;
    \item join is distributive over union;
    \item $\sigma_{false}(R) = \emptyset$ and $\sigma_{true}(R) = R$.
\end{itemize}
Green et al. \cite{green2007provenance} showed that these identities hold for $\mathcal{RA^+}$ on $K$-relations \textit{iff} $(K,+,\cdot,0,1)$ is a \textit{commutative semiring}.
I.e., $(K,+,0)$ and $(K,\cdot,1)$ are commutative monoids\footnotemark,
\footnotetext{An algebraic structure that is closed under an associative binary operation and has an identity element.} 
$\cdot$ is distributive over + and $\forall a, 0 \cdot a = a \cdot 0 = 0$. \\

% Important Cases
\subsection{Important cases} Several important semirings that are discussed in the literature \cite{green2007provenance, Karvounarakis:2012:SDQ:2380776.2380778, Senellart2017} are:
\begin{itemize}
    \item $(\mathbb{B}, \vee, \wedge, false, true)$ - binary trust (set semantics);
    \item $(\mathbb{N}, +, \cdot, 0, 1)$ - multiplicity (bag semantics);
    \item $(\mathbb{A}, min, max, 0, P)$ - security semiring (access control), where the total order $\mathbb{A}=P<C<S<T<0$ describes levels of security clearance: $P$ public, $C$ confidential, $S$ secret, and $T$ top-secret;
    \item $([0,1], max, \cdot, 0, 1)$ - Viterbi semiring (confidence scores, probability);
    % \item $(\mathbb{N}\cup\{\infty\}, min, +, \infty, 0)$ - tropical semiring (data pricing);
    \item $(\mathbb{N}[X], 0, 1, +, \cdot)$ - used for a general form of provenance, the provenance polynomials (the universal semiring). $\mathbb{N}[X]$ is the set of multivariate polynomials with coefficients from $\mathbb{N}$ and variables from a set $X$.
\end{itemize}

\section{ProvSQL - A Real World Provenance Application}\label{sec:provsql}
% Goal
ProvSQL is an open-source project developed by Pierre Senellart et al. \cite{Senellart2018}.
According to the official GitHub page \cite{provsql_github}:
``The goal of the ProvSQL project is to add support for (m-)semiring provenance and uncertainty management to PostgreSQL databases, in the form of a PostgreSQL extension/module/plugin. It is work in progress at the moment." \\
Next, we present several concepts that are incorporated in ProvSQL and briefly discuss its implementation.
% m-semirings
\subsection{Semirings with monus} We previously stated in section \ref{sec:semiring provenance} that provenance semirings extend only to  $\mathcal{RA^+}$ queries. However, \cite{monus_k_relations} identified a large class of semirings that (subject to certain restrictions) can be equipped with a monus operator $-$. Thus, it is possible to generalize provenance capturing to $\mathcal{RA}$ with a difference $(\setminus)$ operation (adding support for non-monotone\footnotemark queries). 
\footnotetext{$I \subseteq I' \notimplies Q(I) \subseteq Q(I')$.}
This class of semirings is called \textit{m-semirings}. Furthermore, \cite{monus_k_relations} show a universal m-semiring, i.e., it is possible to obtain a provenance evaluation in any other m-semiring by applying the appropriate semiring homomorphism.
% Provenance Circuits
\subsection{Provenance Circuits}\label{sec:provenance circuits}
As shown previously by Green et al. \cite{green2007provenance} and Karvounarakis et al. \cite{Karvounarakis:2012:SDQ:2380776.2380778}, provenance explanations can be expressed via semiring formulas (polynomials). These formulas may blow-up in terms of space consumption, and, thus, they are problematic for practical use. An alternative (more compact) representation for provenance annotations is  \textit{circuits} \cite{Deutch2014, Senellart2017}, which are constructed per-query. A provenance circuit is an inductively built directed acyclic graph (DAG), with the following properties:
\begin{itemize}
    \item The leaves contain annotations of tuples from the input DB.
    \item Inner nodes represent operators from a particular semiring (termed \textit{gates} by Senellart et al.). 
    \item The edges (termed \textit{wires} by Senellart et al.) connect nodes to an operator, representing an intermediate calculation.
    \item The sub-DAG under a given node represents the semiring formula for deriving it. 
\end{itemize}
% PostgreSQL Hooks
\subsection{PostgreSQL Hooks} PostgreSQL (Postgres) hooks \cite{postgre_hooks} make it possible to extend/modify its behaviour without rebuilding Postgres, by interrupting the execution process at certain points. Similarly to Postgres itself, the hooks API is written in C. Every hook is accessible via a global function pointer, initially set to \texttt{NULL}. During an extension's loading (following a \texttt{CREATE EXTENSION} command) Postgres calls the extension's own \texttt{\_PG\_init} function (if implemented), which has access to the hooks handler pointers (at this point, a hook function can be registered). When Postgres needs to call a hook, it checks the relevant function pointer, and calls the registered function, if the pointer is set.
% Implementation
\usetikzlibrary{calc,trees,positioning,arrows,chains,shapes.geometric,%
    decorations.pathreplacing,decorations.pathmorphing,shapes,%
    matrix,shapes.symbols}

\tikzset{
>=stealth',
  punktchain/.style={
    rectangle, 
    rounded corners, 
    % fill=black!10,
    draw=black, very thick,
    text width=22em, 
    minimum height=3em, 
    text centered, 
    on chain},
  line/.style={draw, thick, <-},
  element/.style={
    tape,
    top color=white,
    bottom color=blue!50!black!60!,
    minimum width=8em,
    draw=blue!40!black!90, very thick,
    text width=10em, 
    minimum height=3.5em, 
    text centered, 
    on chain},
  every join/.style={->, thick,shorten >=1pt},
  decoration={brace},
  tuborg/.style={decorate},
  tubnode/.style={midway, right=2pt},
}
\begin{figure}[htb]
\small
  \begin{tikzpicture}
  [node distance=.8cm,
  start chain=going below,]
    %  \node[punktchain, fill=yellow!20] (input) {Input query};
     \node[punktchain, join, fill=red!20] (parser) {Query Parser};
     \node[punktchain, join] (A) {Ignore direct references\footnotemark to the \texttt{provsql} column};
     \node[punktchain, join] (B) {Add a new result column `\texttt{provsql}' to the parsed query tree};
     \node[punktchain, join] (C) {Register special SQL functions that will operate on the \texttt{provsql} column during query execution};
    \node[punktchain, join, fill=red!20] (planner) {Query Planner};
    % \node[punktchain, join, fill=green!20] (output) {Query result:\\ includes a \texttt{provsql} column of tokens, identifying gates of the prov. circuit};
  % Now that we have finished the main figure let us add some "after-drawings"
  % Now, let us add some braches. 
  %% No. 1
%   \draw[tuborg, decoration={brace}] let \p1=(init.north), \p2=(init.south) in
%     ($(2, \y1)$) -- ($(2, \y2)$) node[tubnode] {Called once, upon extension loading};
%   %% No. 2
%   \draw[tuborg, decoration={brace}] let \p1=(fini.north), \p2=(fini.south) in
%     ($(2, \y1)$) -- ($(2, \y2)$) node[tubnode] {Called once, upon extension unloading};
  %% No. 3
  \draw[tuborg, decoration={brace}] let \p1=(A.north), \p2=(C.south) in
    ($(4.45, \y1)$) -- ($(4.45, \y2)$) node[tubnode] {\texttt{provsql\_planner\_hook}};
  \end{tikzpicture}
   \textbf{\caption{\label{fig:provsql_architecture}ProvSQL system architecture.\\
   The red rectangles are a part of Postgres's built in execution pipeline. The white rectangles are ProvSQL's implementation of a \texttt{planner\_hook}, which we use for the provenance-related calculations.}}
\normalsize
\end{figure}
\footnotetext{Explicit mentions of the \texttt{provsql} column in the query (in any of the \texttt{SELECT}, \texttt{FROM} or \texttt{WHERE} clauses).}
%%% Local Variables: 
%%% mode: latex
%%% TeX-master: t
%%% End:
\subsection{Implementation} ProvSQL \cite{provsql_github} uses the \texttt{planner\_hook}, which is called after a query has been parsed, and before it is sent to the query planner. The system architecture (as part of Postgres's query execution pipeline) is depicted in Figure \ref{fig:provsql_architecture}.
ProvSQL currently supports a wide range of non-aggregate SQL queries (for more details see \cite{provsql_github, Senellart2018}). The generated query result includes a \texttt{provsql} column of unique\footnotemark
\footnotetext{128-bit universally unique identifiers (UUIDs) that are generated using the \texttt{uuid-ossp} PostgreSQL module.}
tokens, identifying gates of the provenance circuit.


\chapter{Lineage via Embeddings}
\label{chap:lineage_embeddings}

\section{Word Embeddings}\label{sec:word_embeddings_intro}
% The need for word embeddings in NLP
Classic NLP research focused on understanding the structure of text. For example, building dependency-based parse trees \cite{melcuk1988, klein-manning-2004-corpus} that represent the syntactic structure of a sentence via grammatical relations between the words. These approaches did not account for the \textbf{meaning} of words. \textit{Word embedding} aims to encode meanings of words (i.e., semantics), via low dimension (usually, 200-300) real-valued vectors, which can be used to compute the similarity of words as well as test for analogies \cite{DBLP:conf/naacl/MikolovYZ13}. Two of the most influential methods for computing word embeddings are the \textsc{Word2Vec} family of algorithms, by Mikolov et al. \cite{DBLP:journals/corr/abs-1301-3781, DBLP:conf/nips/MikolovSCCD13} and \textsc{GloVe} by Pennington et al. \cite{pennington2014glove}. Furthermore, applying neural-network (NN) techniques to NLP problems (machine translation \cite{wu2016googles}, named entity recognition \cite{Gillick_2016}, sentiment analysis \cite{Maas:2011:LWV:2002472.2002491} etc.) naturally leads to the representation of words and text as real-valued vectors. 

\section{Motivation}\label{sec:motivation_approx}
A few problems become apparent when considering \textit{Distant Provenance} (i.e., indirect, history long, explanations for the existence of tuples in the DB) with traditional and state of the art ``exact provenance tracking" techniques:
\begin{itemize}
    \item Formula based representations (e.g., semiring polynomials \cite{green2007provenance}) may blow-up in terms of space consumption.
    A naive implementation scenario using semiring polynomials requires saving the full provenance polynomial for each tuple. 
    This approach results in a massive growth in space consumption for storing those polynomials (for tuples that are produced by a query and that may depend on result tuples of previous queries).
    \item Inductively built representations (e.g., circuits \cite{Deutch2014, Senellart2017}) would become very complex over time. Thus, they result in impractical provenance querying time. A naive implementation scenario using circuits would simply keep on constructing provenance circuits as described in section \ref{sec:provenance circuits}. During lineage querying, we may end up with very complex circuits, such that numerous leaves are derived via a circuit of their own (these leaves are tuples that were produced by a previous query and were inserted to the DB). This implies that even if a significant amount of sharing is realized across the provenance circuits - these constructions are inevitably going to blow-up in space consumption. This approach renders keeping and querying the full provenance impractical and requires limiting heavily the provenance resolution, otherwise (e.g., return a summarized explanation). 
    \item Complex explanations are not very human-readable. Deutch et al. \cite{Deutch2017} showed how to generate more human-readable explanations - but they are arguably still complex. A ``top-N justifications" style provenance might be more useful for an analyst. 
\end{itemize}


% \section{Proposed Solution}\label{sec:proposed_solution_approx}
% % Our proposal - encode provenance information using word embeddings
% \par We devise a novel approach to provenance tracking, which is based on ML and NLP techniques. The main idea is summarizing, and thereby approximating, the lineage of every tuple with a set of up to $max\_vectors\_num$ (a hyperparameter) vectors. For tuples that are inserted explicitly into the DB, the vectors are obtained using a pre-trained word embeddings model $M$ over the underlying ``text" of the tuple (see Figure \ref{algo:init_approx}).
% During a query execution process we form the lineage of new query result tuples in a similar fashion to that of provenance semirings \cite{green2007provenance}. We extend the + and $\cdot$ operations (see Figures \ref{algo:add_approx} and
% \ref{algo:mul_approx}, respectively) to generate lineage embeddings, while emphasizing the ability to propagate information and preserve the representation of lineage via a set of up to $max\_vectors\_num$, constant-size vectors. We obtain lineage embeddings for query output tuples by using this process. These new tuples (and their lineage) may be later inserted into the DB (depending on the specific application). 
% \par A real world provenance system like ProvSQL \cite{provsql_github} can make use of our lineage vectors during the construction of provenance circuits. Recall that each node in a provenance circuit is associated with some annotation/token, either representing a tuple from the input DB or an intermediate calculation (see section \ref{sec:provsql}). Lineage vectors can be incorporated in such a system by calling the Algorithm in Figure \ref{algo:provsql_online_approx}, for instance. This is an ``online" approach, meaning the lineage vectors are generated during the query execution process. Another approach to incorporating lineage vectors in a real world provenance system might be using such a system as a black box, generating a provenance how-formula for each query output tuple. Later, this formula can be analyzed ``offline", parsed, and converted to lineage vectors by evaluating the parsed expression in a hierarchical manner. Each intermediate evaluation step would call either Figure \ref{algo:add_approx} or \ref{algo:mul_approx}, based on the parsing of the formula, mimicking the online nature of algorithm \ref{algo:provsql_online_approx}.


% \begin{runexample}
%     % Show examples of addition and multiplication constructions
%     Next, we show examples of the addition and multiplication lineage vectors constructions (Figures \ref{algo:add_approx} and
% \ref{algo:mul_approx}, respectively).
%     Let $\vec v_1, \vec v_2, \vec v_3, \vec v_4 \in \mathbb{R}^2$ be vectors, such that:
%     \begin{equation*}
%         \vec v_1 = \begin{pmatrix} -1\\ 0.5 \end{pmatrix},
%         \vec v_2 = \begin{pmatrix} 1\\ 1 \end{pmatrix},
%         \vec v_3 = \begin{pmatrix} -0.5\\ 1 \end{pmatrix},
%         \vec v_4 = \begin{pmatrix} 0\\ -1 \end{pmatrix}
%     \end{equation*}
%     Suppose we have two tuples $t_1, t_2$ with respective sets of lineage vectors $LV_1, LV_2$, such that:
%     \begin{equation*}
%         LV_1 = \{\vec v_1, \vec v_2, \vec v_3\}, LV_2 = \{\vec v_4\}
%     \end{equation*}
%     Finally, the hyperparameters are:
%     \begin{equation*}
%         max\_vectors\_num = 3
%     \end{equation*}
% \end{runexample}
% % addition
% \begin{example-withrun}
%     Let us follow the construction of $LV_3$, which represents the lineage embeddings of $t_1 + t_2$ using the addition Algorithm in Figure \ref{algo:add_approx} (corresponds to alternative use of data, i.e., OR in the query):
%     \begin{enumerate}
%         \item $LV_1 \cup LV_2 = \{\vec v_1, \vec v_2, \vec v_3\} \cup \{\vec v_4\} = \{\vec v_1, \vec v_2, \vec v_3, \vec v_4\}$
%         \item $|\{\vec v_1, \vec v_2, \vec v_3, \vec v_4\}| = 4 > 3 = max\_vectors\_num$
%         \item $LV_3 = \operatorname{ClusterVectorsUsingKMeans}(\{\vec v_1, \vec v_2, \vec v_3, \vec v_4\}) = \{\vec c_1, \vec c_2, \vec c_3\}$\\
%         such that: 
%         \begin{equation*}
%             \vec c_1 = \begin{pmatrix} -0.75\\ 0.75 \end{pmatrix},
%             \vec c_2 = \begin{pmatrix} 1\\ 1 \end{pmatrix},
%             \vec c_3 = \begin{pmatrix} 0\\ -1 \end{pmatrix}
%         \end{equation*}
%         are the centroids of the three clusters.
%     \end{enumerate}
% \end{example-withrun}
% % multiplication
% \begin{example-withrun}
%     Let us follow the construction of $LV_3$, that represents the lineage embeddings of $t_1 \cdot t_2$ using the multiplication Algorithm in Figure \ref{algo:mul_approx} (corresponds to joint use of data, i.e., AND in the query):
%     \begin{enumerate}
%         \item $\operatorname{CartesianProduct}(LV_1, LV_2) = \{(\vec v_1, \vec v_4), (\vec v_2, \vec v_4), (\vec v_3, \vec v_4)\}$
%         \item $\{\operatorname{Avg}(\vec v_1, \vec v_4), \operatorname{Avg}(\vec v_2, \vec v_4), \operatorname{Avg}(\vec v_3, \vec v_4)\} = \{\vec a_1, \vec a_2, \vec a_3\}$\\
%         such that:
%         \begin{equation*}
%             \vec a_1 = \begin{pmatrix} -0.5\\ -0.25 \end{pmatrix},
%             \vec a_2 = \begin{pmatrix} 0.5\\ 0 \end{pmatrix},
%             \vec a_3 = \begin{pmatrix} -0.25\\ 0 \end{pmatrix}
%         \end{equation*}
%         are the average vectors, of each pair, in the cartesian product.
%         \item $|\{\vec a_1, \vec a_2, \vec a_3\}| = 3 \leq 3 = max\_vectors\_num$
%         \item $LV_3 = \{\vec a_1, \vec a_2, \vec a_3\}$
%     \end{enumerate}
% \end{example-withrun}
% % Algorithm - use Algorithm2e
% \begin{figure}[h]
\begin{algorithm2e}[H]
\SetAlgoLined
% \SetKwInOut{Input}{input}\SetKwInOut{Output}{output}
\SetKwFunction{WeightedAverage}{\fncolor{WeightedAverage}}
\SetKwData{vt}{v$_t$}

\KwIn{a pre-trained word embeddings model $M$, an input database $D$}
\KwResult{calculate an initial lineage vector \vt for every tuple $t$ in $D$}
\BlankLine
\BlankLine
 \For{$t \in D.tuples$}{
  \cmtcolor{\tcc{$M$ is a pre-trained word embeddings model and $t$ is a DB tuple}}
  $word\_vectors_t = \{M(w)\mid w \in t\}$\;
  \cmtcolor{\tcc{Perform a "smart" averaging over the word vectors of every word in $t$. That is, average the word vectors for each column separately, and then apply weighted average on the "column vectors".}}%\label{cmt}
  \vt = \WeightedAverage{\paramcolor{$word\_vectors_t$}}\;
}
% \caption{Lineage vectors: initialization}\label{fig:init_approx}
\end{algorithm2e}
\caption{Lineage vectors: Initialization Algorithm}\label{algo:init_approx}
\end{figure}
% % \footnotetext{The square brackets notation [ ] constructs a list, analogous to set construction. The order of words in the list is arbitrary.}
% \begin{algorithm2e}[t]
\SetAlgoLined
% \SetKwInOut{Input}{input}\SetKwInOut{Output}{output}
\SetKwFunction{ClusterVectorsUsingKMeans}{\fncolor{ClusterVectorsUsingKMeans}}
\SetKwData{pv}{LV$_3$}
\SetKwInOut{HP}{Hyper-Parameters}

\HP{\textit{max\_vectors\_num} - maximum number of vectors per-tuple}
\KwIn{two sets of lineage vectors $LV_1, LV_2$ that represent the lineage of two tuples $t_1$ and $t_2$, respectively}
\KwResult{a new set of lineage vectors \pv, that represents $t_1 + t_2$}
\BlankLine
\BlankLine
 \cmtcolor{\tcc{Call this algorithm via $+(LV_1,LV_2)$ or using the infix notation $LV_1 + LV_2$}}
 \BlankLine
 $\pv = LV_1 \cup LV_2$\;
 \If{$|\pv| > max\_vectors\_num$}{
    \cmtcolor{\tcc{Perform K-Means clustering over the vectors into \textit{max\_vectors\_num} groups and return the centers (i.e., each center is a vector) of each group}}%\label{cmt}
    \pv = \ClusterVectorsUsingKMeans{\paramcolor{$\pv$}}\;
 }
\caption{Lineage vectors: $\boldsymbol+$ \small{(addition)}}\label{algo:add_approx}
\end{algorithm2e}
% \begin{algorithm2e}
\SetAlgoLined
% \DontPrintSemicolon
% \SetKwInOut{Input}{input}\SetKwInOut{Output}{output}
\SetKwFunction{ClusterVectorsUsingKMeans}{\fncolor{ClusterVectorsUsingKMeans}}
\SetKwFunction{Avg}{\fncolor{Avg}}
\SetKwFunction{CartesianProduct}{\fncolor{CartesianProduct}}
\SetKwData{pv}{LV$_3$}
\SetKwInOut{HP}{Hyper-Parameters}

\HP{\textit{max\_vectors\_num} - maximum number of vectors per-tuple}
\KwIn{two sets of lineage vectors $LV_1, LV_2$ that represent the lineage of two tuples $t_1$ and $t_2$, respectively}
\KwResult{a new set of lineage vectors \pv, that represents $t_1 \cdot t_2$}
\BlankLine
\BlankLine
 \cmtcolor{\tcc{Call this algorithm via $\cdot(LV_1,LV_2)$ or using the infix notation $LV_1 \cdot LV_2$}}
 \BlankLine
 \pv = $\{\Avg(\paramcolor{v_1, v_2})\mid v_1,v_2 \in \CartesianProduct(\paramcolor{LV_1, LV_2})\}$\;
 \If{$|\pv| > max\_vectors\_num$}{
    \cmtcolor{\tcc{Perform K-Means clustering over the vectors into \textit{max\_vectors\_num} groups and return the centers (i.e., each center is a vector) of each group}}%\label{cmt}
    \pv = \ClusterVectorsUsingKMeans{\paramcolor{$\pv$}}\; 
 }
\caption{Lineage vectors: $\boldsymbol\cdot$ \small{(multiplication)}}\label{algo:mul_approx}
\end{algorithm2e}
% \begin{algorithm2e}[t]
\SetAlgoLined
% \SetKwInOut{Input}{input}\SetKwInOut{Output}{output}
% \SetKwFunction{ClusterVectorsUsingKMeans}{\fncolor{ClusterVectorsUsingKMeans}}
\SetKwData{token}{token$_3$}
% \SetKwInOut{HP}{Hyper-Parameters}

% \HP{\textit{max\_vectors\_num} - maximum number of vectors per-tuple}
\KwIn{a pair of tokens $token_1, token_2$ 
and their associated sets of lineage vectors $LV_1, LV_2$\newline
an operation $op \in \{+, \cdot\}$}
\KwResult{a new token \token, that is associated with the new set of lineage vectors $LV_3$ = $LV_1\:op\:LV_2$}
\BlankLine
\BlankLine
 \BlankLine
 \Switch{$op$}{
    \Case{$+$}{
        $LV_3$ = $LV_1 + LV_2$ \cmtcolor{\tcp*{see Algorithm \ref{algo:add_approx}}}
    }
    \Case{$\cdot$}{
        $LV_3$ = $LV_1 \cdot LV_2$ \cmtcolor{\tcp*{see Algorithm \ref{algo:mul_approx}}}
    }
 }
 tag the newly generated $LV_3$ as \token\;
\caption{Incorporating lineage vectors in ProvSQL}\label{algo:provsql_online_approx}
\end{algorithm2e}
% % \begin{algorithm2e}[t]
\SetAlgoLined
% \SetKwInOut{Input}{input}\SetKwInOut{Output}{output}
% \SetKwFunction{ClusterVectorsUsingKMeans}{\fncolor{ClusterVectorsUsingKMeans}}
\SetKwData{pv}{LV$_3$}
% \SetKwInOut{HP}{Hyper-Parameters}

% \HP{\textit{max\_vectors\_num} - maximum number of vectors per-tuple}
\KwIn{two sets of lineage vectors $LV_1, LV_2$ that represent the lineage of two tuples $t_1$ and $t_2$, respectively}
\KwResult{a new set of lineage vectors \pv, that represents $t_1 + t_2$}
\BlankLine
\BlankLine
 \cmtcolor{\tcc{Call this algorithm via $+(LV1,LV2)$ or using the infix notation $LV1 + LV2$}}
 \BlankLine
%  $\pv = LV_1 \cup LV_2$\;
%  \If{$|\pv| > max\_vectors\_num$}{
%     \cmtcolor{\tcc{Perform K-Means clustering over the vectors into \textit{max\_vectors\_num} groups and return the centers (i.e., each center is a vector) of each group}}%\label{cmt}
%     \pv = \ClusterVectorsUsingKMeans{\paramcolor{$\pv$}}\;
%  }
\caption{Convert provenance formula to lineage vectors}\label{algo:formula_offline_approx}
\end{algorithm2e}


% % \footnotetext{Perform K-Means clustering over the vectors into \textit{max\_vectors\_num} groups and return the centers of each group.}

% % Explain the motivation and process of approx provenance querying

% \subsection{Lineage querying} As was mentioned in section \ref{sec:word_embeddings_intro}, word embeddings (i.e., vectors) provide insights into the meaning of words via a word-word similarity score. The similarity measure is the cosine distance between the word vectors. Our goal is to construct and maintain lineage embeddings, which can provide insights to the reasons for the existence of tuples via a tuple-tuple similarity score, which is analogous to word-word similarity score (the tuples of interest are the ones generated by a query, which in turn use previously query-generated tuples as well as explicitly inserted ones).
% \par Consequently, given a tuple and its lineage embeddings, we can calculate the pair-wise similarity against every other tuple in the DB (or a subset, e.g., in a specific table) and return the top $N$ (a parameter) most lineage-similar tuples (these resemble a subset of the lineage \cite{Cui:2000:TLV:357775.357777}). 
% There are many algorithms for approximate vector search, e.g., based on LSH \cite{lsh}. Approximate vector search is a very active area of research and we can utilize known algorithms (see, e.g., \cite{sugawara-etal-2016-approximately}).
% Due to the reliance of lineage embeddings on an underlying statistical ML model, we expect our produced lineage to \textbf{approximate} the exact \textbf{lineage}.\\

% % \footnotetext{Call this algorithm via $+(LV1,LV2)$ or using the infix notation $LV1 + LV2$.}
% % \footnotetext{Call this algorithm via $\cdot(LV1,LV2)$ or using the infix notation $LV1 \cdot LV2$.}

% % indtended usage clarification
% \subsection{Intended usage} The intended usage of lineage vectors is as follows:
% \begin{itemize}
%     \item Each manually inserted tuple - has a set consisting of a single tuple vector.
%     \item Each tuple in the result of a query - has a set of up to $max\_vectors\_num$ tuple vectors.
%     \item When calculating similarity - we always compare between two sets of lineage vectors, by using the formula we shall present in section \ref{sec:latent_wv_model}.
% \end{itemize}
% % Show how our approach addresses the problems listed in the motivation section
% \subsection{Comparison to exact lineage} In section \ref{sec:motivation_approx} we listed some problems that arise when approaching distant (i.e., over the whole DB history) lineage computation with traditional techniques. Our proposed solution addresses these problems as follows:
% \begin{itemize}
%     \item Lineage embeddings require only a constant space per-tuple.
%     \item Lineage embeddings are immutable once computed, and do not get more complex over time (querying time depends only on the number of tuples, with which we compare the target tuple, i.e., the one to be explained via lineage).
%     \item Returned explanations (produced lineage) are simply the top-N justifications, and may assist in further analysis.
% \end{itemize}
% We approximate lineage using embedding techniques. As noted, lineage is a type of provenance, and it can be expressed using semirings. Now, semirings can also express how-provenance, which is more general and informative than lineage, but, it becomes extremely complex to track as histories develop. Thus, lineage seems to be a more practical tool for analysts.
% A major advantage of our approach is realized for distant provenance. A process that is equivalent functionally to recursive drill through is done automatically (i.e., if we envision the distant provenance of a tuple as a recursive tree-like structure).\\

% \par After developing the concept of lineage embeddings in isolation, it is essential to test it as part of a real-world implementation. We developed a Python module that adds ``lineage via embeddings" capabilities to a Postgres DBMS, by integrating our module with the ProvSQL \cite{Senellart2018} extension. The construction of the lineage vectors is based on the algorithms presented above.


\section{Word Vectors Model}\label{sec:latent_wv_model}
% All logic and vector training from db code is written in python (detached from ProvSQL, enabling faster development)
Now, let us discuss a component which is crucial to the success of our system - the word vectors model. \\
% Word Vectors Training
\subsection{Training word vectors} 
As in Bordawekar et al. \cite{DBLP:journals/corr/BordawekarS16}, we train a \textsc{Word2Vec} model \cite{rehurek_lrec} on a corpus that is extracted from the relevant DB.
A naive transformation of a DB to unstructured text (a sequence of sentences) can be achieved by simply concatenating the textual representation of the different columns of each tuple into a separate sentence. This approach has several problems \cite{DBLP:journals/corr/BordawekarS16}:
\begin{itemize}
    \item When dealing with natural language text, there is an assumption that the semantic influence of a word on a nearby word is inversely proportional to the distance between them. However, not only that a sentence extracted from a tuple does not necessarily correspond to any natural language structure, but, it can be actually thought of as ``a bag of columns"; i.e., the order between different columns in a sentence has no semantic implications.
    \item All columns are not the same. That is, some columns may hold more semantic importance than others in the same sentence (generated from a tuple). For instance, a primary key column, a foreign key column, or an important domain-specific column (e.g., a manufacturer column in a products table). This implies that in order to derive meaningful embeddings from the trained word vectors, we need to consider inter-column discrimination, during both training (i.e., when constructing initial lineage vectors for every tuple in the DB - see Figure \ref{algo:init_approx}) and evaluation phases.
\end{itemize}
We solve these problems by properly setting hyperparameters (e.g., window size), artificially (and carefully) injecting repeated text inside a sentence, and dividing the training into multiple, different stages. 
A detailed outline of the ``textification" and training processes follows:
\begin{enumerate}
    \item Generate a \texttt{Key} column with unique values for each table in the DB.
    \item Extract two corpora from the DB, one with columns (except the \texttt{Key} columns) as sentences, and the other with tuples as sentences.
    \item Transform each numerical value in the generated corpora by concatenating the relevant column name to the number. Other methods of dealing with numerical values exist as well, and are discussed in chapter \ref{chap:discussion}.
    \item Transform each sentence in the generated corpora by injecting the corresponding \texttt{Key} value repeatedly after every $k$ (a separate hyperparameter for the columns corpus and the tuples corpus) words in the sentence. The idea is to encode more information about each tuple in its corresponding \texttt{Key} value vector.
    \item Incrementally train a \textsc{Word2Vec} model on the generated corpora. First, train on the columns corpus and then on the tuples corpus. One may question why we have the same model for both the tuple-based (see chapter \ref{chap:per_tuple_lineage_vectors}) and the column-based (see chapter \ref{chap:per_column_lineage_vectors}) lineage vectors. The answer is that experimentally the models that are restricted to only the tuples corpus or the columns corpus gave inferior results.
\end{enumerate}
One more interesting caveat of dealing with texts extracted from relations, is that missing data (empty/null cells) is sometimes automatically converted (when an external tool, e.g., Python, is used to interface with PostgreSQL) to `None'. Given that `None' is not a \textit{stop word} (words that are removed from a corpus as part of a common practice of cleanup before training), we realized that `None' adds noise to the vectors (if there is a lot of missing data in the DB). Adding `None' to the list of stop words improved drastically both the word embedding model quality and the overall accuracy of our system. \\

\par The quality of the word vectors model is crucial to the success of our system. However, optimizing the overall performance should focus not only on the training phase, but also on the way we utilize the trained model. Next, we show a number of such optimizations.\\
% sentence embeddings
\subsection{Sentence embeddings} Extracting sentence embeddings from text has been a well-researched topic in the NLP community over the last five years. State-of-the-art pre-trained models (e.g., Universal Sentence Encoder \cite{Cer2018UniversalSE, use_github} and BERT \cite{devlin2018bert}) are trained on natural language texts, and thus are not suitable for sentences generated from relational tuples (see discussion above). Hence we train a word embedding model and infer the sentence vectors as a \textit{function of the set of word vectors containing all the words} in a sentence. We average the word vectors for each column separately, and then apply weighted average on the ``column vectors" (the weight is based on the relative importance of a column, as discussed above). As will be shown, column-based vectors result in significant improvements to lineage encoding.
% comparing two sets of vectors
\subsection{Similarity calculation} Given two word vectors, the similarity score is usually the cosine distance between them. In our system, we want to calculate the similarity between the lineage representations of two tuples/columns (see chapters \ref{chap:per_tuple_lineage_vectors} and \ref{chap:per_column_lineage_vectors}, respectively); they each have a \textit{set} of lineage vectors. That is, we need to calculate the similarity between \textit{two sets of vectors}\footnotemark. The logic behind the following formula is balancing between the ``best pair" of vectors (in terms of similarity) in the two sets and the similarity between their average vectors:\\
\begin{equation*}
    \operatorname{sim}(A, B) = \frac{w_{max} \cdot max(ps) + w_{avg} \cdot avg(ps)}
    {w_{max} + w_{avg}}\\
\end{equation*}
\footnotetext{We note that \cite{
DBLP:journals/corr/BordawekarS16, 
DBLP:journals/corr/abs-1712-07199,
DBLP:conf/sigmod/BordawekarS17}
have also used various similar methods for comparing two sets of vectors.}
where $ps$ is the set of pair-wise similarities between a pair of vectors, one taken from set $A$ and one taken from set $B$. $w_{max}$ and $w_{avg}$ are (user-specified) hyperparameters. $max$ and $avg$ are functions that return the maximum and average values of a collection of numbers, respectively\footnotemark.
\footnotetext{This logic holds for both tuple-based vectors and column-based vectors (i.e., for each column separately).}


\section{Related Work}
% Relational Embeddings
\textbf{Relational embedding} is a very active area of research \cite{DBLP:journals/corr/abs-1803-01384, sigmod2020_keynote, pie2020_keynote}.
When converting a relational DB to unstructured text (see section \ref{sec:latent_wv_model}) special care is required to support numerical values correctly. Bordawekar and Shmueli \cite{DBLP:journals/corr/BordawekarS16} take the route of tokenizing numerical values from the DB by preceding each number with a ``range designator" (e.g., 1-10, 50-100, SMALL, BIG, etc.) in the generated corpus. In \cite{DBLP:journals/corr/abs-1712-07199} Bordawekar, Bandyopadhyay and Shmueli use clustering techniques to represent numerical values textually. In contrast, in this work we deal with numerical values by concatenating the relevant column name to each number in the generated corpus. This way, we ensure our model separates between numerical values of different fields, as they are conceptually different, semantically speaking (for example, we want the value 300 for a \texttt{nutrient\_code} field in a \texttt{nutrients} table to have a different learned vector than the value 300 for a \texttt{household\_serving\_size} field in a \texttt{serving\_size} table, from the USDA BFPDB \cite{usda_bfpd-dataset} dataset). 


% \section{Interim Findings and Hurdles}
% We conducted a series of experiments (see chapter \ref{chap:experimental_evaluation}) and came to the following interim conclusions:
% \begin{itemize}
%     \item Our explanations encode the text and not the tuples. Thus, relations with many similar tuples (text-wise) result in our system performing poorly. We assume that most of the problems listed below are symptoms of the same root cause.
%     \item Analysis of direct provenance (i.e., \textit{direct} explanations for the existence of tuples in a query result in terms of existing DB tuples, regardless of their historical raison d'être) on simple, complex (ones that might involve multiple tables with non-trivial joins and non-trivial ``where" conditions) and composite (that contain sub-queries) queries gives good results ($> 90\%$ accuracy for ``top $50\%$'' of the tuples in the lineage) generally. However, results suffer from inconsistencies sometimes, identifying tuples incorrectly as part of the linaege due to thier textual similarity to actual lineage tuples.
%     \item Distant provenance provides useful results. However, it suffers from the same problem of textual closeness.
% \end{itemize}

% \par In the next chapter we introduce methods that are designed to overcome these defficiencies.
\include{main/conclusion}
%
% Add any appendices here; they must come _before_ the bibliography
%
\appendix
%\noappendicestocpagenum
%\addappheadtotoc
\chapter{BFPDB Materialized Views}
\label{appendix:bfpdb_materiazlied_views}

Following is a detailed description of the materialized views (used in the experiments in section \ref{sec:advanced_experiments}) and the queries that created them.

\section{\texttt{prepared}}\label{appendix:sec:prepared}
Contains all the products-related information from the \texttt{products} table, for products that have a \textit{prepared} preparation state in the \texttt{serving\_size} table:
\newcommand{\sqlcolor}[1]{\textcolor{Bittersweet}{#1}}
\newcommand{\fieldcolor}[1]{\textcolor{Fuchsia}{#1}}
\newcommand{\strcolor}[1]{\textcolor{ForestGreen}{#1}}
\newcommand{\numcolor}[1]{\textcolor{TealBlue}{#1}}

% \begin{adjustbox}{width=\columnwidth,center}
\begin{figure}[h]
% \raggedright
% \begin{small}
    \texttt{
    % \textcolor{white}{\tiny something}\\
    \textcolor{white}{F}\sqlcolor{SELECT} p.\fieldcolor{ndb\_no}, p.\fieldcolor{manufacturer}, p.\fieldcolor{name}, p.\fieldcolor{ingredients} \\
    \textcolor{white}{FF}\sqlcolor{FROM} products p, serving\_size ss\\
    \textcolor{white}{FF}\sqlcolor{WHERE} p.\fieldcolor{ndb\_no} = ss.\fieldcolor{ndb\_no} \\ 
    \textcolor{white}{FF}\sqlcolor{AND} ss.\fieldcolor{preparation\_state} = \strcolor{'prepared'}
    }
    % \textbf{\caption{\label{fig:queries_usda_bpfd}Running example query for the USDA Branded Products Food Database}}
% \end{small}
\end{figure}
% \end{adjustbox}

\section{\texttt{unprepared}}\label{appendix:sec:unprepared}
Contains all the products-related information from the \texttt{products} table, for products that have a \textit{unprepared} preparation state in the \texttt{serving\_size} table:
% \begin{adjustbox}{width=\columnwidth,center}
\begin{figure}[h]
% \raggedright
% \begin{small}
    \texttt{
    % \textcolor{white}{\tiny something}\\
    \textcolor{white}{F}\sqlcolor{SELECT} p.\fieldcolor{ndb\_no}, p.\fieldcolor{manufacturer}, p.\fieldcolor{name}, p.\fieldcolor{ingredients} \\
    \textcolor{white}{FF}\sqlcolor{FROM} products p, serving\_size ss\\
    \textcolor{white}{FF}\sqlcolor{WHERE} p.\fieldcolor{ndb\_no} = ss.\fieldcolor{ndb\_no} \\ 
    \textcolor{white}{FF}\sqlcolor{AND} ss.\fieldcolor{preparation\_state} = \strcolor{'unprepared'}
    }
    % \textbf{\caption{\label{fig:queries_usda_bpfd}Running example query for the USDA Branded Products Food Database}}
% \end{small}
\end{figure}
% \end{adjustbox}

\section{\texttt{readytodrink}}\label{appendix:sec:readytodrink}
Contains all the products-related information from the \texttt{products} table, for products that have a \textit{readytodrink} preparation state in the \texttt{serving\_size} table:
% \begin{adjustbox}{width=\columnwidth,center}
\begin{figure}[h]
% \raggedright
% \begin{small}
    \texttt{
    % \textcolor{white}{\tiny something}\\
    \textcolor{white}{F}\sqlcolor{SELECT} p.\fieldcolor{ndb\_no}, p.\fieldcolor{manufacturer}, p.\fieldcolor{name}, p.\fieldcolor{ingredients} \\
    \textcolor{white}{FF}\sqlcolor{FROM} products p, serving\_size ss\\
    \textcolor{white}{FF}\sqlcolor{WHERE} p.\fieldcolor{ndb\_no} = ss.\fieldcolor{ndb\_no} \\ 
    \textcolor{white}{FF}\sqlcolor{AND} ss.\fieldcolor{preparation\_state} = \strcolor{'readytodrink'}
    }
    % \textbf{\caption{\label{fig:queries_usda_bpfd}Running example query for the USDA Branded Products Food Database}}
% \end{small}
\end{figure}
% \end{adjustbox}


\section{\texttt{protein}}\label{appendix:sec:protein}
Contains all the products-related information from the \texttt{products} table, for products that have a \textit{protein} nutrient-related information in the \texttt{nutrients} table:
% \begin{adjustbox}{width=\columnwidth,center}
\begin{figure}[H]
% \raggedright
% \begin{small}
    \texttt{
    % \textcolor{white}{\tiny something}\\
    \textcolor{white}{F}\sqlcolor{SELECT} p.\fieldcolor{ndb\_no}, p.\fieldcolor{manufacturer}, p.\fieldcolor{name}, p.\fieldcolor{ingredients} \\
    \textcolor{white}{FF}\sqlcolor{FROM} products p, nutrients n\\
    \textcolor{white}{FF}\sqlcolor{WHERE} p.\fieldcolor{ndb\_no} = n.\fieldcolor{ndb\_no} \\ 
    \textcolor{white}{FF}\sqlcolor{AND} n.\fieldcolor{nutrient\_name} = \strcolor{'protein'}
    }
    % \textbf{\caption{\label{fig:queries_usda_bpfd}Running example query for the USDA Branded Products Food Database}}
% \end{small}
\end{figure}
% \end{adjustbox}


\section{\texttt{sugars}}\label{appendix:sec:sugars}
Contains all the products-related information from the \texttt{products} table, for products that have a \textit{sugars} nutrient-related information in the \texttt{nutrients} table:
% \begin{adjustbox}{width=\columnwidth,center}
\begin{figure}[H]
% \raggedright
% \begin{small}
    \texttt{
    % \textcolor{white}{\tiny something}\\
    \textcolor{white}{F}\sqlcolor{SELECT} p.\fieldcolor{ndb\_no}, p.\fieldcolor{manufacturer}, p.\fieldcolor{name}, p.\fieldcolor{ingredients} \\
    \textcolor{white}{FF}\sqlcolor{FROM} products p, nutrients n\\
    \textcolor{white}{FF}\sqlcolor{WHERE} p.\fieldcolor{ndb\_no} = n.\fieldcolor{ndb\_no} \\ 
    \textcolor{white}{FF}\sqlcolor{AND POSITION}(\strcolor{'sugars'} \sqlcolor{IN} n.\fieldcolor{nutrient\_name}) > \numcolor{0}
    }
    % \textbf{\caption{\label{fig:queries_usda_bpfd}Running example query for the USDA Branded Products Food Database}}
% \end{small}
\end{figure}
% \end{adjustbox}


\section{\texttt{cholesterol}}\label{appendix:sec:cholesterol}
Contains all the products-related information from the \texttt{products} table, for products that have a \textit{cholesterol} nutrient-related information in the \texttt{nutrients} table:
% \begin{adjustbox}{width=\columnwidth,center}
\begin{figure}[H]
% \raggedright
% \begin{small}
    \texttt{
    % \textcolor{white}{\tiny something}\\
    \textcolor{white}{F}\sqlcolor{SELECT} p.\fieldcolor{ndb\_no}, p.\fieldcolor{manufacturer}, p.\fieldcolor{name}, p.\fieldcolor{ingredients} \\
    \textcolor{white}{FF}\sqlcolor{FROM} products p, nutrients n\\
    \textcolor{white}{FF}\sqlcolor{WHERE} p.\fieldcolor{ndb\_no} = n.\fieldcolor{ndb\_no} \\ 
    \textcolor{white}{FF}\sqlcolor{AND POSITION}(\strcolor{'cholesterol'} \sqlcolor{IN} n.\fieldcolor{nutrient\_name}) > \numcolor{0}
    }
    % \textbf{\caption{\label{fig:queries_usda_bpfd}Running example query for the USDA Branded Products Food Database}}
% \end{small}
\end{figure}
% \end{adjustbox}


\section{$\texttt{exp}\sb\texttt{1}$}\label{appendix:sec:exp1}
Contains distinct manufacturers of products that have \textit{sugar} as an ingredient, and contain \textit{protein} nutrient information. Also, these manufacturers produce \textit{prepared} products: 
% \begin{adjustbox}{width=\columnwidth,center}
\begin{figure}[H]
% \raggedright
% \begin{small}
    \texttt{
    % \textcolor{white}{\tiny something}\\
    \textcolor{white}{F}\sqlcolor{SELECT} prt.\fieldcolor{manufacturer} \\
    \textcolor{white}{FF}\sqlcolor{FROM} protein prt, (\sqlcolor{SELECT DISTINCT} \fieldcolor{manufacturer} \sqlcolor{FROM} prepared) t\\
    \textcolor{white}{FF}\sqlcolor{WHERE POSITION}(\strcolor{'sugar'} \sqlcolor{IN} prt.\fieldcolor{ingredients}) > \numcolor{0} \\
    \textcolor{white}{FF}\sqlcolor{AND} prt.\fieldcolor{manufacturer} = t.\fieldcolor{manufacturer}\\
    \textcolor{white}{FF}\sqlcolor{GROUP BY} prt.\fieldcolor{manufacturer}
    }
    % \textbf{\caption{\label{fig:queries_usda_bpfd}Running example query for the USDA Branded Products Food Database}}
% \end{small}
\end{figure}
% \end{adjustbox}


\section{$\texttt{exp}\sb\texttt{2}$}\label{appendix:sec:exp2}
Contains distinct manufacturers of products that have \textit{water} as an ingredient, and contain \textit{protein} nutrient information. Also, these manufacturers produce \textit{unprepared} products: 
% \begin{adjustbox}{width=\columnwidth,center}
\begin{figure}[H]
% \raggedright
% \begin{small}
    \texttt{
    % \textcolor{white}{\tiny something}\\
    \textcolor{white}{F}\sqlcolor{SELECT} prt.\fieldcolor{manufacturer} \\
    \textcolor{white}{FF}\sqlcolor{FROM} protein prt, (\sqlcolor{SELECT DISTINCT} \fieldcolor{manufacturer} \sqlcolor{FROM} unprepared) t\\
    \textcolor{white}{FF}\sqlcolor{WHERE POSITION}(\strcolor{'water'} \sqlcolor{IN} prt.\fieldcolor{ingredients}) > \numcolor{0} \\
    \textcolor{white}{FF}\sqlcolor{AND} prt.\fieldcolor{manufacturer} = t.\fieldcolor{manufacturer}\\
    \textcolor{white}{FF}\sqlcolor{GROUP BY} prt.\fieldcolor{manufacturer}
    }
    % \textbf{\caption{\label{fig:queries_usda_bpfd}Running example query for the USDA Branded Products Food Database}}
% \end{small}
\end{figure}
% \end{adjustbox}


\section{$\texttt{exp}\sb\texttt{3}$}\label{appendix:sec:exp3}
Contains distinct manufacturers of products that have \textit{water} and \textit{sugar} as an ingredient, and contain \textit{protein} nutrient information. Also, these manufacturers produce \textit{prepared} and \textit{unprepared} products:
% \begin{adjustbox}{width=\columnwidth,center}
\begin{figure}[H]
% \raggedright
% \begin{small}
    \texttt{
    % \textcolor{white}{\tiny something}\\
    \textcolor{white}{F}\sqlcolor{SELECT} $\texttt{t}\sb\texttt{2}$.\fieldcolor{manufacturer} \sqlcolor{FROM}\\
    \textcolor{white}{FFFF}(\sqlcolor{SELECT} prt.\fieldcolor{manufacturer} \\
    \textcolor{white}{FFFFF}\sqlcolor{FROM} protein prt, (\sqlcolor{SELECT DISTINCT} \fieldcolor{manufacturer} \sqlcolor{FROM} prepared) t\\
    \textcolor{white}{FFFFF}\sqlcolor{WHERE POSITION}(\strcolor{'sugar'} \sqlcolor{IN} prt.\fieldcolor{ingredients}) > \numcolor{0} \\
    \textcolor{white}{FFFFF}\sqlcolor{AND} prt.\fieldcolor{manufacturer} = t.\fieldcolor{manufacturer}\\
    \textcolor{white}{FFFFF}\sqlcolor{GROUP BY} prt.\fieldcolor{manufacturer}) $\texttt{t}\sb\texttt{1}$,\\
    \textcolor{white}{FFFF}(\sqlcolor{SELECT} prt.\fieldcolor{manufacturer} \\
    \textcolor{white}{FFFFF}\sqlcolor{FROM} protein prt, (\sqlcolor{SELECT DISTINCT} \fieldcolor{manufacturer} \sqlcolor{FROM} unprepared) t\\
    \textcolor{white}{FFFFF}\sqlcolor{WHERE POSITION}(\strcolor{'water'} \sqlcolor{IN} prt.\fieldcolor{ingredients}) > \numcolor{0} \\
    \textcolor{white}{FFFFF}\sqlcolor{AND} prt.\fieldcolor{manufacturer} = t.\fieldcolor{manufacturer}\\
    \textcolor{white}{FFFFF}\sqlcolor{GROUP BY} prt.\fieldcolor{manufacturer}) $\texttt{t}\sb\texttt{2}$,\\
    \textcolor{white}{FF}\sqlcolor{WHERE} $\texttt{t}\sb\texttt{1}$.\fieldcolor{manufacturer} = $\texttt{t}\sb\texttt{2}$.\fieldcolor{manufacturer}
    }
    % \textbf{\caption{\label{fig:queries_usda_bpfd}Running example query for the USDA Branded Products Food Database}}
% \end{small}
\end{figure}
% \end{adjustbox}


\section{$\texttt{exp}\sb\texttt{4}$}\label{appendix:sec:exp4}
Contains 10 distinct pairs of (product identifier, nutrient derivation code) for products that are produced by manufacturers from the $\texttt{exp}\sb\texttt{1}$ \textbf{or} $\texttt{exp}\sb\texttt{2}$ materialized views:
% \begin{adjustbox}{width=\columnwidth,center}
\begin{figure}[H]
% \raggedright
% \begin{small}
    \texttt{
    % \textcolor{white}{\tiny something}\\
    \textcolor{white}{F}\sqlcolor{SELECT} n.\fieldcolor{ndb\_no}, n.\fieldcolor{derivation\_code} \\
    \textcolor{white}{FF}\sqlcolor{FROM} $\texttt{exp}\sb\texttt{1}$, $\texttt{exp}\sb\texttt{2}$, products p, nutrients n\\
    \textcolor{white}{FF}\sqlcolor{WHERE} ($\texttt{exp}\sb\texttt{1}$.\fieldcolor{manufacturer} = p.\fieldcolor{manufacturer}\\
    \textcolor{white}{FFFFFF}\sqlcolor{OR} $\texttt{exp}\sb\texttt{2}$.\fieldcolor{manufacturer} = p.\fieldcolor{manufacturer}) \\
    \textcolor{white}{FF}\sqlcolor{AND} p.\fieldcolor{ndb\_no} = n.\fieldcolor{ndb\_no}\\
    \textcolor{white}{FF}\sqlcolor{GROUP BY} n.\fieldcolor{ndb\_no}, n.\fieldcolor{derivation\_code} \sqlcolor{LIMIT} \numcolor{10}
    }
    % \textbf{\caption{\label{fig:queries_usda_bpfd}Running example query for the USDA Branded Products Food Database}}
% \end{small}
\end{figure}
% \end{adjustbox}


% Back Matter
% ------------

% The following command will typeset the bibliography,
% then typeset the Hebrew part of the thesis:
% - Cover page
% - Title page
% - Acknowledgements page
%  (NO table of contents or list of figures in Hebrew)
% - (Extended) abstract (1000-2000 words)
%
% based on information you've provided in the thesis-fields file
% (including the relative paths to your bib files). The Hebrew
% content will be typeset in _reverse_page_order_, i.e. first
% in the file will be the last page of the abstract, and the
% Hebrew cover page will be the last page of the file.
%
\makebackmatter

% The resulting PDF can be printed and taken straight to binding,
% i.e. you do not need to flip any pages anywhere. Of course,
% mind the LaTeX error and warning messages, overfull hboxes etc.

\end{document}

