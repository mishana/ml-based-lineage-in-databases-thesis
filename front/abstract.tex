% This file contains the abstract part of your thesis - in English and
% in Hebrew (within \abstractEnglish and \abstractHebrew respectively).
%
% Notes:
% - This file uses the UTF-8 character set encoding for the Hebrew
%   text not to get garbled. Keep it that way.
% - Assuming your thesis is mainly in English, Graduate School 
%   regulations mandate the following lengths for the abstracts:
%
%      Language    Min. Length   Max. Length
%     ---------------------------------------
%      English       200 words     500 words
%      Hebrew      1,000 words   2,000 words
%
%   so that the Hebrew abstract typically has some content from
%   the English introduction and an overview of the results, not
%   present in the English; it is not just a translation.

\abstractEnglish{

In recent years, there has been extensive research on \textit{data provenance}. Previous works were concerned with annotating the results of database (DB) queries with provenance information. This information proved useful in explaining query results at various resolution levels.
\par In this work, we aspire to track the lineage of tuples throughout their database lifetime. That is, we consider a scenario in which tuples that are produced by a query may affect other tuple insertions into the DB, as part of a normal workflow. As time goes on, provenance explanations for such tuples become deeply nested and highly complex in terms of space consumption, lineage querying time, clarity and readability.
\par We present a novel approach for \textit{approximating} lineage tracking. We use Machine Learning (ML) and Natural Language Processing (NLP) techniques; mainly, \textit{word embedding}. We integrate our lineage computations into PostgreSQL via an extension (ProvSQL) and exhibit useful results in terms of accuracy against exact, semiring-based justifications. We argue that our proposed solution does not suffer from space complexity blow-up over time, and thus makes a provenance tracking system in such a setting feasible. A significant benefit of our approach is a ``natural ranking" of explanations.


} % end of English abstract


\abstractHebrew{

% Note that certain commands don't work that well in Hebrew "mode".
% If this happens to you, try wrapping the command within a
% \textenglish{ } - that may (or may not) help.

יוחסין של נתונים הוא נושא שנחקר רבות בשנים האחרונות.
עבודות קודמות עסקו בתיוג של תוצאות שאילתות על גבי מסדי נתונים על ידי ׳׳מטא נתונים״ המעידים על היוחסין של התוצאות.
מטא נתונים אלה הוכחו כשימושיים בהסבר של תוצאות שאילתות ברמות פירוט שונות.

במחקר זה אנו שואפים להתחקות אחר המוצאות של רשומות במסדי נתונים לאורך חייהן.
כלומר, אנחנו מניחים תסריט שבו רשומות הנוצרות על ידי שאילתא עלולות להשפיע על הכנסתן של רשומות אחרות למסד הנתונים, וזאת כחלק מתזרים עבודה תקני של המערכת.
עם חלוף הזמן, אילנות היוחסין (וההסברים הנובעים מהם) של רשומות אלה הופכים מקוננים ומסובכים מאוד מבחינת צריכת זיכרון, זמן חישוב עבור מוצאות רשומות, בהירות וקריאות. 

אנו מציגים גישה חדשה לחישוב מקורב של מוצאות רשומות תוך שימוש בטכניקות של למידת מכונה ועיבוד שפה טבעית (ובעיקר, שיכון מילים במרחב אוקלידי).
אנו משלבים את החישוב המקורב של מוצאות רשומות במערכת לניהול מסדי נתונים
\textenglish{PostgreSQL}
באמצעות ההרחבה
\textenglish{ProvSQL}
ומציגים תוצאות בעלות תועלת מבחינת דיוק בהשוואה לחישוב מדויק המבוסס חצאי-חוגים (מבנה אלגברי).
אנו טוענים כי הפיתרון המוצע אינו סובל מניפוח יתר בסיבוכיות מקום לאורך זמן,
ולכן הופך מערכת לחקר יוחסין של רשומות ברת ביצוע תחת האילוצים שתוארו לעיל.
יתרון משמעותי של השיטה שלנו הוא ״דירוג טבעי״ של מוצאות רשומות.


% בית הספר ללימודי מוסמכים מנחה מספר הנחיות לגבי התקציר בעברית:
% \begin{itemize}
% \item על התקציר להיכתב במשפטים מקושרים שלמים.
% \item בדרך-כלל אין לציין בתקציר מקורות ספרותיים וציטוטים.
% \item אין להתייחס למספר של פרק, סעיף, נוסחה, ציור או טבלה שבגוף החיבור, ואין להשתמש בקיצורים, סמלים ומונחים לא מקובלים, אלא אם יש בתקציר די מקום לזיהויים.
% \end{itemize}



% \subsection*{\texthebrew{תת-חלק בתקציר המורחב}}

% תוכן מקוצר לגבי נושא מסוים. התייחסות ל\emph{מושג} מסוים שהחיבור בוחן. וכולי וכולי.


% \subsection*{\texthebrew{נקודה מעניינת לגבי העמודים בעברית}}

% שימו לב כי העמודים בעברית אמורים להיות מיוצרים בסדר ה''הפוך'', הווה אומר העמוד האחרון בקובץ ה-\textenglish{PDF} הוא הכריכה העברית, לפניו השער העברי, ודפי התקציר צריכים להופיע בסדר הפוך (וכן במספור רומי, לפי נהלי הטכניון). כך אם נתבונן במספר שבתחתית עמוד זה \textenglish{---} אשר צריך להיות העמוד הראשון בתקציר-המורחב מבחינת רצף התוכן, והינו העמוד האחרון מבין עמודי התקציר-המורחב אחרון בקובץ ה-\textenglish{PDF} \textenglish{---} נמצא את המספר \textenglish{i} ...

% \newpage

% ... ואילו עמוד זה של התקציר-המורחב בעברית \textenglish{---} שהינו העמוד השני בתקציר-המורחב מבחינת רצף התוכן, ונמצא ראשון בקובץ ה-\textenglish{PDF} \textenglish{---} ממוספר ב-\textenglish{ii}. המטרה במספור בסדר ה"הפוך" היא, שבעת ההדפסה לא יהיה צורך להפוך דפים, לשנות את סדרם וכולי \textenglish{---} רק להדפיס ולכרוך.

} % end of Hebrew abstract
