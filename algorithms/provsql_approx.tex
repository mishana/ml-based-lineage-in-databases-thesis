\begin{figure}[h]
\begin{algorithm2e}[H]
\SetAlgoLined
% \SetKwInOut{Input}{input}\SetKwInOut{Output}{output}
% \SetKwFunction{ClusterVectorsUsingKMeans}{\fncolor{ClusterVectorsUsingKMeans}}
\SetKwData{token}{token$_3$}
% \SetKwInOut{HP}{Hyper-Parameters}

% \HP{\textit{max\_vectors\_num} - maximum number of vectors per-tuple}
\KwIn{a pair of tokens $token_1, token_2$ 
and their associated sets of lineage vectors $LV_1, LV_2$\newline
an operation $op \in \{+, \cdot\}$}
\KwResult{a new token \token, that is associated with the new set of lineage vectors $LV_3$ = $LV_1\:op\:LV_2$}
\BlankLine
\BlankLine
 \BlankLine
 \Switch{$op$}{
    \Case{$+$}{
        $LV_3$ = $LV_1 + LV_2$ \cmtcolor{\tcp*{see Algorithm \ref{algo:add_approx}}}
    }
    \Case{$\cdot$}{
        $LV_3$ = $LV_1 \cdot LV_2$ \cmtcolor{\tcp*{see Algorithm \ref{algo:mul_approx}}}
    }
 }
 tag the newly generated $LV_3$ as \token\;
\end{algorithm2e}
\caption{Incorporating lineage vectors in ProvSQL Algorithm}\label{algo:provsql_online_approx}
\end{figure}