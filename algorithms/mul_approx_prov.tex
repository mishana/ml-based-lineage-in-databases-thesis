\begin{figure}[h]
\begin{algorithm2e}[H]
\SetAlgoLined
% \DontPrintSemicolon
% \SetKwInOut{Input}{input}\SetKwInOut{Output}{output}
\SetKwFunction{ClusterVectorsUsingKMeans}{\fncolor{ClusterVectorsUsingKMeans}}
\SetKwFunction{Avg}{\fncolor{Avg}}
\SetKwFunction{CartesianProduct}{\fncolor{CartesianProduct}}
\SetKwData{pv}{LV$_3$}
\SetKwInOut{HP}{Hyper-Parameters}

\HP{\textit{max\_vectors\_num} - maximum number of vectors per-tuple}
\KwIn{two sets of lineage vectors $LV_1, LV_2$ that represent the lineage of two tuples $t_1$ and $t_2$, respectively}
\KwResult{a new set of lineage vectors \pv, that represents $t_1 \cdot t_2$}
\BlankLine
\BlankLine
 \cmtcolor{\tcc{Call this algorithm via $\cdot(LV_1,LV_2)$ or using the infix notation $LV_1 \cdot LV_2$}}
 \BlankLine
 \pv = $\{\Avg(\paramcolor{v_1, v_2})\mid v_1,v_2 \in \CartesianProduct(\paramcolor{LV_1, LV_2})\}$\;
 \If{$|\pv| > max\_vectors\_num$}{
    \cmtcolor{\tcc{Perform K-Means clustering over the vectors into \textit{max\_vectors\_num} groups and return the centers (i.e., each center is a vector) of each group}}%\label{cmt}
    \pv = \ClusterVectorsUsingKMeans{\paramcolor{$\pv$}}\; 
 }
\end{algorithm2e}  
\caption{Lineage vectors: $\boldsymbol\cdot$ \small{(multiplication) Algorithm}}\label{algo:mul_approx}
\end{figure}
