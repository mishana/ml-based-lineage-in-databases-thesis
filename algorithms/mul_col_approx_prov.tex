\begin{algorithm2e}
\SetAlgoLined
% \DontPrintSemicolon
% \SetKwInOut{Input}{input}\SetKwInOut{Output}{output}
\SetKwFunction{ClusterVectorsUsingKMeans}{\fncolor{ClusterVectorsUsingKMeans}}
\SetKwFunction{Avg}{\fncolor{Avg}}
\SetKwFunction{CartesianProduct}{\fncolor{CartesianProduct}}
\SetKwData{pv}{CV$_3$}
\SetKwInOut{HP}{Hyper-Parameters}

% \HP{\textit{max\_vectors\_num} - maximum number of vectors per-tuple}
\KwIn{two maps of: \textit{columns $\rightarrow$ sets of lineage vectors} $CV_1, CV_2$ that represent the lineage of two tuples $t_1$ and $t_2$, respectively}
\KwResult{a new map of: \textit{columns $\rightarrow$ sets of lineage vectors} \pv, that represents $t_1 \cdot t_2$}
\BlankLine
\BlankLine
 \pv = \{\} $\rightarrow$ \{\}\cmtcolor{\tcp*{an empty map of: \textit{columns $\rightarrow$ sets of lineage vectors}}}
 \ForEach{column $c \in (CV_1.columns \cup CV_2.columns)$}{
    \cmtcolor{\tcc{$CV_i.columns$ is the domain and $CV_i[c]$ is the corresponding \textit{set of lineage vectors} of tuple $t_i$ for the column $c$}}
    \uIf{$c \in (CV_1.columns \cap CV_2.columns)$}{
        $\pv[c]$ = $\paramcolor{CV_1[c]} \:\fncolor{\boldsymbol\cdot}\: \paramcolor{CV_2[c]}$\cmtcolor{\tcp*{see Algorithm \ref{algo:mul_approx}}}
        % \cmtcolor{\tcp*{an inline call to $\boldsymbol\cdot$ operation defined in Algorithm \ref{algo:mul_approx}}}
    }
    \uElseIf{$c \in CV_1.columns$}{
        $\pv[c]$ = $CV_1[c]$\;
    }
    \Else{
        $\pv[c]$ = $CV_2[c]$\;
    }
 }
\caption{Column Lineage vectors: $\boldsymbol\cdot$ \small{(multiplication)}}\label{algo:mul_col_approx}
\end{algorithm2e}